\documentclass[12pt,]{article}
\usepackage[left=1in,top=1in,right=1in,bottom=1in]{geometry}
\newcommand*{\authorfont}{\fontfamily{phv}\selectfont}
\usepackage[]{mathpazo}


  \usepackage[T1]{fontenc}
  \usepackage[utf8]{inputenc}




\usepackage{abstract}
\renewcommand{\abstractname}{}    % clear the title
\renewcommand{\absnamepos}{empty} % originally center

\renewenvironment{abstract}
 {{%
    \setlength{\leftmargin}{0mm}
    \setlength{\rightmargin}{\leftmargin}%
  }%
  \relax}
 {\endlist}

\makeatletter
\def\@maketitle{%
  \newpage
%  \null
%  \vskip 2em%
%  \begin{center}%
  \let \footnote \thanks
    {\fontsize{18}{20}\selectfont\raggedright  \setlength{\parindent}{0pt} \@title \par}%
}
%\fi
\makeatother




\setcounter{secnumdepth}{0}

\usepackage{color}
\usepackage{fancyvrb}
\newcommand{\VerbBar}{|}
\newcommand{\VERB}{\Verb[commandchars=\\\{\}]}
\DefineVerbatimEnvironment{Highlighting}{Verbatim}{commandchars=\\\{\}}
% Add ',fontsize=\small' for more characters per line
\usepackage{framed}
\definecolor{shadecolor}{RGB}{248,248,248}
\newenvironment{Shaded}{\begin{snugshade}}{\end{snugshade}}
\newcommand{\AlertTok}[1]{\textcolor[rgb]{0.94,0.16,0.16}{#1}}
\newcommand{\AnnotationTok}[1]{\textcolor[rgb]{0.56,0.35,0.01}{\textbf{\textit{#1}}}}
\newcommand{\AttributeTok}[1]{\textcolor[rgb]{0.77,0.63,0.00}{#1}}
\newcommand{\BaseNTok}[1]{\textcolor[rgb]{0.00,0.00,0.81}{#1}}
\newcommand{\BuiltInTok}[1]{#1}
\newcommand{\CharTok}[1]{\textcolor[rgb]{0.31,0.60,0.02}{#1}}
\newcommand{\CommentTok}[1]{\textcolor[rgb]{0.56,0.35,0.01}{\textit{#1}}}
\newcommand{\CommentVarTok}[1]{\textcolor[rgb]{0.56,0.35,0.01}{\textbf{\textit{#1}}}}
\newcommand{\ConstantTok}[1]{\textcolor[rgb]{0.00,0.00,0.00}{#1}}
\newcommand{\ControlFlowTok}[1]{\textcolor[rgb]{0.13,0.29,0.53}{\textbf{#1}}}
\newcommand{\DataTypeTok}[1]{\textcolor[rgb]{0.13,0.29,0.53}{#1}}
\newcommand{\DecValTok}[1]{\textcolor[rgb]{0.00,0.00,0.81}{#1}}
\newcommand{\DocumentationTok}[1]{\textcolor[rgb]{0.56,0.35,0.01}{\textbf{\textit{#1}}}}
\newcommand{\ErrorTok}[1]{\textcolor[rgb]{0.64,0.00,0.00}{\textbf{#1}}}
\newcommand{\ExtensionTok}[1]{#1}
\newcommand{\FloatTok}[1]{\textcolor[rgb]{0.00,0.00,0.81}{#1}}
\newcommand{\FunctionTok}[1]{\textcolor[rgb]{0.00,0.00,0.00}{#1}}
\newcommand{\ImportTok}[1]{#1}
\newcommand{\InformationTok}[1]{\textcolor[rgb]{0.56,0.35,0.01}{\textbf{\textit{#1}}}}
\newcommand{\KeywordTok}[1]{\textcolor[rgb]{0.13,0.29,0.53}{\textbf{#1}}}
\newcommand{\NormalTok}[1]{#1}
\newcommand{\OperatorTok}[1]{\textcolor[rgb]{0.81,0.36,0.00}{\textbf{#1}}}
\newcommand{\OtherTok}[1]{\textcolor[rgb]{0.56,0.35,0.01}{#1}}
\newcommand{\PreprocessorTok}[1]{\textcolor[rgb]{0.56,0.35,0.01}{\textit{#1}}}
\newcommand{\RegionMarkerTok}[1]{#1}
\newcommand{\SpecialCharTok}[1]{\textcolor[rgb]{0.00,0.00,0.00}{#1}}
\newcommand{\SpecialStringTok}[1]{\textcolor[rgb]{0.31,0.60,0.02}{#1}}
\newcommand{\StringTok}[1]{\textcolor[rgb]{0.31,0.60,0.02}{#1}}
\newcommand{\VariableTok}[1]{\textcolor[rgb]{0.00,0.00,0.00}{#1}}
\newcommand{\VerbatimStringTok}[1]{\textcolor[rgb]{0.31,0.60,0.02}{#1}}
\newcommand{\WarningTok}[1]{\textcolor[rgb]{0.56,0.35,0.01}{\textbf{\textit{#1}}}}
\usepackage{longtable,booktabs}



\title{Political Donor Motivations and Social Media: A Time-Series Analysis  }



\author{\Large \vspace{0.05in} \newline\normalsize\emph{}  }


\date{}

\usepackage{titlesec}

\titleformat*{\section}{\normalsize\bfseries}
\titleformat*{\subsection}{\normalsize\itshape}
\titleformat*{\subsubsection}{\normalsize\itshape}
\titleformat*{\paragraph}{\normalsize\itshape}
\titleformat*{\subparagraph}{\normalsize\itshape}





\newtheorem{hypothesis}{Hypothesis}
\usepackage{setspace}


% set default figure placement to htbp
\makeatletter
\def\fps@figure{htbp}
\makeatother

\usepackage{graphicx}

% move the hyperref stuff down here, after header-includes, to allow for - \usepackage{hyperref}

\makeatletter
\@ifpackageloaded{hyperref}{}{%
\ifxetex
  \PassOptionsToPackage{hyphens}{url}\usepackage[setpagesize=false, % page size defined by xetex
              unicode=false, % unicode breaks when used with xetex
              xetex]{hyperref}
\else
  \PassOptionsToPackage{hyphens}{url}\usepackage[draft,unicode=true]{hyperref}
\fi
}

\@ifpackageloaded{color}{
    \PassOptionsToPackage{usenames,dvipsnames}{color}
}{%
    \usepackage[usenames,dvipsnames]{color}
}
\makeatother
\hypersetup{breaklinks=true,
            bookmarks=true,
            pdfauthor={ ()},
             pdfkeywords = {},  
            pdftitle={Political Donor Motivations and Social Media: A Time-Series Analysis},
            colorlinks=true,
            citecolor=blue,
            urlcolor=blue,
            linkcolor=magenta,
            pdfborder={0 0 0}}
\urlstyle{same}  % don't use monospace font for urls

% Add an option for endnotes. -----


% add tightlist ----------
\providecommand{\tightlist}{%
\setlength{\itemsep}{0pt}\setlength{\parskip}{0pt}}

% add some other packages ----------

% \usepackage{multicol}
% This should regulate where figures float
% See: https://tex.stackexchange.com/questions/2275/keeping-tables-figures-close-to-where-they-are-mentioned
\usepackage[section]{placeins}


\begin{document}
	
% \pagenumbering{arabic}% resets `page` counter to 1 
%
% \maketitle

{% \usefont{T1}{pnc}{m}{n}
\setlength{\parindent}{0pt}
\thispagestyle{plain}
{\fontsize{18}{20}\selectfont\raggedright 
\maketitle  % title \par  

}

{
   \vskip 13.5pt\relax \normalsize\fontsize{11}{12} 
\textbf{\authorfont } \hskip 15pt \emph{\small }   

}

}






\vskip -8.5pt


 % removetitleabstract

\noindent \doublespacing 

The two predominant theories of political donor motivations are the
access-oriented model (Fouirnaies and Hall 2015) and the consumption
model (Ansolabehere, Figueiredo, and Snyder 2003). In the
access-oriented model, individual political donors and political action
committees (PACs) contribute to campaigns in an effort to acquire access
and influence politicians into supporting specific policy issues
(Fouirnaies and Hall 2015). The consumption model of donors sees
political contributions as being an extension of voting along a
participatory spectrum, and that donors support candidates who they
already know support policy issues that the donors care about or are
ideologically motivated (Barber 2016; Johnson 2010). Previous studies
have posited these two models of political donor motivations against
each other (Heerwig 2016).

This paper combines political donation records and Twitter and Facebook
posts from politicians in the 2016 election cycle. These two datasets
are analyzed in conjunction with one another to see if politicians'
public support of various policy issues on social media precede, lag, or
have no relationship with donations from various communities of
political donors. The access-oriented model of political donors predicts
that donations from specific groups of donors would precede public
support of certain policies. In contrast, the consumption model of
political donors predicts that donations from various groups of donors
would lag in response to public support of certain policy issues by
candidates. This paper sets these two models of political donors against
each to see if evidence of either model is found in observational data.

\hypertarget{data}{%
\section{Data}\label{data}}

Data for this research comes from two primary sources: politicians'
social media posts and political donation data. For social media posts,
this paper used the Facebook and Twitter APIs to collect social media
posts from all candidates for the Wisconsin State Senate and Wisconsin
State Assembly during the 2016 election cycle (\emph{n} = 82,851). A
subset of these posts were hand-coded into 27 topical categories. This
subset was used to train a BERT deep learning transfer model that was
used to predict the topic of the remainder of the posts (training
dataset = 8,242, 10\% of total posts; testing dataset = 4,122, 5\% of
total posts). Political donation data for all candidates to the
Wisconsin State Legislature during the 2016 election cycle were
collected from the Wisconsin Campaign Information System (CFIS). These
donations were used to create a network of political donations with
candidates and donors serving as nodes and donations between them as
edges. This network was clustered into distinct communities so that
donors in each community are most similar to one another based on which
campaigns they contributed to. I theorize that these clusters/
communities represent different coalitions of political donors with
unique motivations. Viewing political donors as members of a coalition
of fundraisers has been studied in the past (Adams 2007; Heerwig 2016)
but often in a more informal sense and not as statistically-derived
groups like this paper.

\hypertarget{methodology}{%
\section{Methodology}\label{methodology}}

These two datasets were analyzed against each other using the granger
causality time-series methodology. This methodology detects whether
movements in one time series precedes, lags, or is not related to
another time series. Specifically, this paper compares time series of
donations from clusters of political donors and times series of the
number of social media posts by each topic that were made by campaigns
that each donor cluster contributed to. For example, a time series of
donations from a donor cluster was compared to the aggregate count of
posts about a given topic made by candidates that the donor cluster
contributed to.

\hypertarget{results}{%
\section{Results}\label{results}}

Initial results suggest that both models of political donor
motivation--the access-oriented and consumptive models--are present in
the observational data. Different clusters of political donors exhibit
different behaviors. Some clusters of political donors show behavior
that is in-line with the access-oriented model, where their donations
precede public support of policy issues by the candidates that they
contributed to. Contributions from other donor clusters align with the
conumption model of political donors and contributions from their
clusters lag public support of various policy issues by candidates.
Other donor clusters exhibit no behavioral connection to candidates'
public support of policy issues--suggesting that there is some
confounding factor or motivation for donating to the campaigns that they
contributed to.

The next step is to analyze the results to see which donor clusters fall
under which donor model and with what policy topics. Another step is to
analyze potential confounding factors that could describe the
motivations of political donors, such as geographic proximity, or
competitiveness of the races the cluster contributed to in aggregate.

\begin{Shaded}
\begin{Highlighting}[]
\NormalTok{wordcountaddin}\OperatorTok{:::}\KeywordTok{text_stats}\NormalTok{()}
\end{Highlighting}
\end{Shaded}

\begin{longtable}[]{@{}lll@{}}
\toprule
Method & koRpus & stringi\tabularnewline
\midrule
\endhead
Word count & 738 & 735\tabularnewline
Character count & 4936 & 4936\tabularnewline
Sentence count & 33 & Not available\tabularnewline
Reading time & 3.7 minutes & 3.7 minutes\tabularnewline
\bottomrule
\end{longtable}

\hypertarget{refs}{}
\leavevmode\hypertarget{ref-adams2016}{}%
Adams, Brian E. 2007. ``Fundraising Coalitions in Open Seat Mayoral
Elections.'' \emph{Journal of Urban Affairs} 29 (5): 481--99.
\url{https://doi.org/10.1111/j.1467-9906.2007.00361.x}.

\leavevmode\hypertarget{ref-ansolabehere2003}{}%
Ansolabehere, Stephen, John M. de Figueiredo, and James M. Snyder. 2003.
``Why Is There so Little Money in U.s. Politics.'' \emph{Journal of
Economic Perspectives} 17 (1): 105--30.

\leavevmode\hypertarget{ref-barber2016}{}%
Barber, Michael. 2016. ``Donation Motivations: Testing Theories of
Access and Ideology.'' \emph{Political Research Quarterly} 69 (1):
148--59.

\leavevmode\hypertarget{ref-fouirnaies2015}{}%
Fouirnaies, Alexander, and Andrew Hall. 2015. ``The Exposure Theory of
Access: Why Some Firms Seek More Access to Incumbents Than Others.''
\emph{SSRN Electronic Journal}, January.
\url{https://doi.org/10.2139/ssrn.2652361}.

\leavevmode\hypertarget{ref-heerwig2016}{}%
Heerwig, Jennifer A. 2016. ``Donations and Dependence: Individual
Contributor Strategies in House Elections.'' \emph{Social Science
Research} 60: 181--98.
\url{https://doi.org/https://doi.org/10.1016/j.ssresearch.2016.06.001}.

\leavevmode\hypertarget{ref-johnson2010}{}%
Johnson, Bertram. 2010. ``Individual Contributions: A Fundraising
Advantage for the Ideologically Extreme?'' \emph{American Politics
Research}, June, 890--908.
\url{https://doi.org/10.1177/1532673X09357500}.





\newpage
\singlespacing 
\end{document}
