\documentclass[12pt,]{article}
\usepackage[left=1in,top=1in,right=1in,bottom=1in]{geometry}
\newcommand*{\authorfont}{\fontfamily{phv}\selectfont}
\usepackage[]{mathpazo}


  \usepackage[T1]{fontenc}
  \usepackage[utf8]{inputenc}




\usepackage{abstract}
\renewcommand{\abstractname}{}    % clear the title
\renewcommand{\absnamepos}{empty} % originally center

\renewenvironment{abstract}
 {{%
    \setlength{\leftmargin}{0mm}
    \setlength{\rightmargin}{\leftmargin}%
  }%
  \relax}
 {\endlist}

\makeatletter
\def\@maketitle{%
  \newpage
%  \null
%  \vskip 2em%
%  \begin{center}%
  \let \footnote \thanks
    {\fontsize{18}{20}\selectfont\raggedright  \setlength{\parindent}{0pt} \@title \par}%
}
%\fi
\makeatother




\setcounter{secnumdepth}{0}

\usepackage{longtable,booktabs}

\usepackage{graphicx,grffile}
\makeatletter
\def\maxwidth{\ifdim\Gin@nat@width>\linewidth\linewidth\else\Gin@nat@width\fi}
\def\maxheight{\ifdim\Gin@nat@height>\textheight\textheight\else\Gin@nat@height\fi}
\makeatother
% Scale images if necessary, so that they will not overflow the page
% margins by default, and it is still possible to overwrite the defaults
% using explicit options in \includegraphics[width, height, ...]{}
\setkeys{Gin}{width=\maxwidth,height=\maxheight,keepaspectratio}


\title{Political Donor Motivations and Public Support of Policies: A Time
Series Analysis  }



\author{\Large \vspace{0.05in} \newline\normalsize\emph{}  }


\date{}

\usepackage{titlesec}

\titleformat*{\section}{\normalsize\bfseries}
\titleformat*{\subsection}{\normalsize\itshape}
\titleformat*{\subsubsection}{\normalsize\itshape}
\titleformat*{\paragraph}{\normalsize\itshape}
\titleformat*{\subparagraph}{\normalsize\itshape}





\newtheorem{hypothesis}{Hypothesis}
\usepackage{setspace}


% set default figure placement to htbp
\makeatletter
\def\fps@figure{htbp}
\makeatother

\usepackage{graphicx}

% move the hyperref stuff down here, after header-includes, to allow for - \usepackage{hyperref}

\makeatletter
\@ifpackageloaded{hyperref}{}{%
\ifxetex
  \PassOptionsToPackage{hyphens}{url}\usepackage[setpagesize=false, % page size defined by xetex
              unicode=false, % unicode breaks when used with xetex
              xetex]{hyperref}
\else
  \PassOptionsToPackage{hyphens}{url}\usepackage[draft,unicode=true]{hyperref}
\fi
}

\@ifpackageloaded{color}{
    \PassOptionsToPackage{usenames,dvipsnames}{color}
}{%
    \usepackage[usenames,dvipsnames]{color}
}
\makeatother
\hypersetup{breaklinks=true,
            bookmarks=true,
            pdfauthor={ ()},
             pdfkeywords = {},  
            pdftitle={Political Donor Motivations and Public Support of Policies: A Time
Series Analysis},
            colorlinks=true,
            citecolor=blue,
            urlcolor=blue,
            linkcolor=magenta,
            pdfborder={0 0 0}}
\urlstyle{same}  % don't use monospace font for urls

% Add an option for endnotes. -----


% add tightlist ----------
\providecommand{\tightlist}{%
\setlength{\itemsep}{0pt}\setlength{\parskip}{0pt}}

% add some other packages ----------

% \usepackage{multicol}
% This should regulate where figures float
% See: https://tex.stackexchange.com/questions/2275/keeping-tables-figures-close-to-where-they-are-mentioned
\usepackage[section]{placeins}


\begin{document}
	
% \pagenumbering{arabic}% resets `page` counter to 1 
%
% \maketitle

{% \usefont{T1}{pnc}{m}{n}
\setlength{\parindent}{0pt}
\thispagestyle{plain}
{\fontsize{18}{20}\selectfont\raggedright 
\maketitle  % title \par  

}

{
   \vskip 13.5pt\relax \normalsize\fontsize{11}{12} 
\textbf{\authorfont } \hskip 15pt \emph{\small }   

}

}








\begin{abstract}

    \hbox{\vrule height .2pt width 39.14pc}

    \vskip 8.5pt % \small 

\noindent The two predominant theories of political donor motivations are the
access-oriented model and the consumption model. This paper combines
political donation records and social media posts from politicians to
test whether either behavior is observed. In the access-oriented model,
individual political donors and political action committees (PACs) are
assumed to contribute to campaigns in an effort to acquire access and
influence politicians into supporting specific policy issues. In this
study, the access-oriented model of donors predicts that donations from
specific groups of donors will precede public support of certain
policies. The consumption model of donors views political contributions
as being an extension of voting along a participatory spectrum, and that
donors support candidates who they already know support policy issues
that the donors care about or are ideologically motivated. In this
research, the consumption model predicts that donations from various
groups of donors will lag in response to public support of certain
policy issues. Historically, these two models have treated political
donors as all having the same motivations. More recent studies in
campaign finance have found that both motivational models can exist in
different groups of donors. However, these studies categorize groups of
donors in broad strokes, generally as either small-dollar donors and
large-dollar donors as well as PACs. This paper statistically derives
coalitions of similar donors and tests the competing models of political
donor motivations on these more granular groups of donors who support
similar candidates.


    \hbox{\vrule height .2pt width 39.14pc}


\end{abstract}


\vskip -8.5pt


 % removetitleabstract

\noindent \doublespacing 

\begin{longtable}[]{@{}lll@{}}
\toprule
Method & koRpus & stringi\tabularnewline
\midrule
\endhead
Word count & 4448 & 4468\tabularnewline
Character count & 30634 & 30625\tabularnewline
Sentence count & 194 & Not available\tabularnewline
Reading time & 22.2 minutes & 22.3 minutes\tabularnewline
\bottomrule
\end{longtable}

\newpage

\hypertarget{introduction}{%
\section{Introduction}\label{introduction}}

Previous studies on the motivations of political donors have often
treated donors as a monolith. These studies would contrast the two
predominant theories of political donor motivation--access-oriented or
consumption--against each other. In the seminal paper, ``Why is There so
Little Money in U.S. Politics?,'' Ansolabehere, de Figueiredo, and
Snyder (2003) stated that ``two extreme views bracket the range of
thinking about the amount of money in U.S. political campaigns.'' They
conclude that ``campaign contributions should be viewed \emph{primarily}
as a type of consumption good.'' Similarly, Gordon, Hager, and Landa
(2007) succinctly asked, ``Consumption or Investment?'' Welch explained
that there two models of campaign contributions. In one model,
contributions are given in exchange for political favors (the
access-oriented model) or to alter election probabilities in a way that
helps one's preferred candidate (the consumption model) (Welch 1980).
Many other studies have focused on a single model of donor motivation
and seek evidence for that model without considering alternatives.

Recently, a more nuanced view of the motivations of political donors has
emerged that sees donors as unique actors. Different donors may have
different motivations, intents, and goals in making a political
contribution. Barber (2016) examines the differences in motivations
between Political Action Committees (PACs) and individual donors,
finding that PACs exhibit behavior inline with the access-oriented model
and individuals appear to be motivated by political consumption/
ideology. Heerwig (2016) drills down into individual donors by
categorizing individual donors as either frequent or infrequent donors,
concluding that frequent donors are more access-oriented whereas
infrequent donors are more motivated by consumption/ ideology. Rhodes
et. al (2018) find four types of unique individual donors:
Party-Oriented Donors, Local-Oriented Donors, Idiosyncratic Donors, and
Nationalized Donors, with each group having a unique set of motivations.
This paper continues in this direction of segmenting donors to
understand unique motivations. Instead of making a descriptive
distinction between donors such PACs versus individuals (Barber 2016),
frequent versus infrequent donors (Heerwig 2016), or other heuristics
(Rhodes, Schaffner, and Raja 2018), this paper clusters donors who are
similar to one another using a network approach where donors in a
cluster are similar to one another in who they contributed to. This
approach is similar to that of Wahl and Sheppard (2018) who then use
other pieces of information from the contributions themselves to
identify the associations that connect the members of the community to
each other. This study takes the approach another step forward by
layering in another unique dataset, social media data from politicians,
to identify behaviors, and potentially the underlying reasons why donors
are in the same statistical community. This paper combines these two
datasets to test theories of motivations in clusters of political
donors.

This network-based approach conceives of clusters of donors acting as a
\emph{coalition} where coalition can have a distinct motivation.
Previous network studies have concluded that this type of network
clustering has been highly predictive for other types of political
analysis, including voting behavior in the U.S. House of Representatives
and Senate (Wahl, Sheppard, and Shanahan 2019). Instead of focusing on
the clusters in the campaign finance networks that legislators belong
to, this study examines the donors themselves and their
statistically-derived clusters. Further, I believe that this
coalition-based approach can capture evidence of donor motivations more
accurately. For example, a contributor may only make a single donation.
That single donation may not have a statistically-identifiable influence
on politicians, but if that single donor is acting in concert with other
donors, there is a potential for identifiable results. For example,
under an access-oriented/ influence model of donor motivations, this
single donor could be a member of a pro-environment interest group. If
many members of this group give individually to a candidate, the
coalition could exhibit influence over that campaign to become more
pro-environment. Similarly, under a consumption/ ideology model of
motivation, a candidate could come out with a strong pro-environmental
message and many members of the coalition could be attracted to the
message and reward the pro-environment stance taken by the campaign.

In addition, this paper adds a dimension of policy issues. Are there
certain policy issues that have donors who exhibit behavior inline with
the access-oriented model or consumption model? It is possible that
different issues are related to different coalitions of donors with
different motivations. For example, perhaps pro-environmental donors are
driven by the access-oriented model and anti-abortion groups are driven
by the consumption model of politics. This paper combines donation
records with social media data collected in Wisconsin during the 2016
election cycle to measure whether campaigns' support of certain policy
issues respond to donations from clusters or whether donations from
coalitions respond to public support of policy issues. Previous studies
have used social media posts as a proxy for public appeals in support of
President Trump. This proxy for public appeals was used to study the
impact of legislators' posts about President Trump on their fundraising
(Fu and Howell 2020). Particularly with the rise of politics online,
adding in social media data provides a valuable variable in
understanding the information ecology that political donors experience
and how the information ecology relates to their donation motivations.

\hypertarget{access-oriented-model}{%
\section{Access-Oriented Model}\label{access-oriented-model}}

Access-oriented political donors are those that attempt to use their
contributions to gain access to politicians. Most often, access-oriented
motivations are thought to be the reason behind contributions from
Political Action Committees (PACs) and donors with business interest.
The theory goes that this access can then influence legislative behavior
(Francia et al. 2003). Milbrath (1958) centers legislative influence as
a communicative process where those seeking to influence legislators
must be able to ``communicate their power, as well as the facts and
arguments supporting their position, when they confer with a
legislator.'' Congress is operating in a ``vacuum filled with noise.''
Political contributions can gain direct access that allows one to cut
through all the noise of competing information that the legislator might
be encountering (Milbrath 1958). In interviews, business groups
themselves said that they seek ``access'' to either a member of congress
or a member of their staff when they make a contribution. But these
groups stated that their contribution only gains the access to make an
argument and it is the merit of the argument that determines support for
their cause or not (Herndon 1982). Surveys confirm that donors to U.S.
House and Senate campaigns try to influence politics in a way that helps
their businesses (Baker 2020a).Empirical studies of financial documents
backup this claim that donors seek access in order to influence policy
(Fouirnaies and Hall 2015).

Past research has suggested that political contributors are successful
in their goals to gain access as measured by the amount of time that
organized interest groups spend with members of congress (Langbein
1986). This increased access has been found not only for PACs but also
for increased access for individual donors. A randomized experiment
found that when it is revealed to congressional offices that prospective
meeting attendees contributed to the member's campaign that senior
members of the office took meetings between three to four times more
often than when information about contributions were withheld (Kalla and
Broockman 2016).

However, measuring the direct access that political financiers gain from
political contributions is difficult to measure. Instead, researchers
have treated the ``access'' component of contributor influence as an
implicit assumption and instead look for evidence of ``influence'' of
political contributors on politicians. Many political science papers do
not use the explicit term ``access-oriented donor'' and instead refer to
their work as examining the potential ``influence'' of political donors
on politicians. This line of influence research inherently implies a
gain of access by political contributors. As Langbein (1986) states,
``{[}Access{]} is a precondition for having influence over public
policy. Contributions themselves have little meaning for a congressman,
because they do not carry any `message.' Only access, or some other form
of direct or indirect communication, can translate money into
influence.''

Even though research has suggested there is a connection between
political contributions and access. It is unclear if that access
actually converts to \emph{influence} in the political process. Despite
confirmation that PACs attempt to influence the legislative process
(Grenzke 1989), past research has found PAC contributions to have a
limited effect on roll-call voting (Wright 1985). In rare instances,
there is an apparent connection between PAC contributions and roll-call
votes, but that correlation is most likely due to broader support from
larger interest groups (Grenzke 1989). These sparse correlations could
be manifestation of the finding that legislators are responsive to
changes in the opinions of the national individual donor class
(Canes-Wrone and Gibson 2019). One article went so far as to conclude
that ``evidence in the article undermines belief in the
military-industrial complex model'' (Wayman 1985) when studying the
effect of defense-related PACs on roll-call voting. One study using
meta-analytic methods found over 93 studies that "corporate political
activity only weakly impacts public policy and at best has a (direct)
weak effect on corporate outcomes (Hadani, Bonardi, and Dahan 2017) that
may be more of a signalling of policy preference than anything else
(Austen-Smith 1995). This signalling is likely only effective if the
contribution is large enough that influence the likelihood of the
candidate being elected (Schnakenberg and Turner 2021).

However, other research has found evidence that contributions from the
financial (Hayes 2017), telecommunications (Edwards and Figueiredo
2016), education (Constant 2006), environmental (Hogan 2020) and
healthcare interest groups (McKay 2018) have influenced legislation
passed. One study on the connection between campaign contributions and
legislative voting does support the idea that moneyed interests play a
significant role in the legislative process, particularly organized
business interests that are within a member's district (Hall and Wayman
1990), potentially similar to how members of congress prioritize public
opinion of their district over national public opinion (Butler and
Nickerson 2011). Further, there appears to be a stronger influence as a
result of contributions from individuals with business interests,
opposed to PACs, which many other studies focus on (Fellowes and Wolf
2004). While it is ``nearly universal'' (Bonica 2016) that corporate
executives of Fortune 500 firms make political contributions, and there
is a significant increase in contributions once the business people are
promoted to executive status (Fremeth, Richter, and Schaufele 2013),
there is heterogeneity in their political leanings (Bonica 2016).

In addition, PACs have been found to use contributions in an attempt to
acquire access and favor, a result that suggests that they ``at least
anticipate that the donations will influence policy'' (Powell and
Grimmer 2016). A meta-analysis found that model specification played a
significant role in whether significant results were found when looking
for a connection between donations and roll-calls votes, concluding that
studies that controlled for ``friendly giving by including a measure of
legislators' ideology and that include more than one contributions
variable are less likely to produce significant results'' (Roscoe and
Jenkins 2005). Despite this variability in model specification, the
authors conclude that one-third of roll-call votes are impacted by
campaign contributions (Roscoe and Jenkins 2005). Research suggests that
campaign contributions only try to influence a small number of votes
that have an outsized impact on whether legislation actually passes or
not (Gordon 2001) or on ideological or highly visible political issues
(Witko 2006).

One consideration when assessing the impact of PAC donations on
roll-call voting is that donations are just a piece of the broader
lobbying effort when trying to influence legislation. Ideologically
extreme groups and groups that are more liberal have been found to favor
PAC contribution over other lobbying methods (McKay 2010). Legislators'
perceptions of the power of the interest group, for example, union
membership rates (Finger 2019) may also be a factor in whether
contributions can acquire influence. Firms that make larger political
contributions have been found to get more ``sweetheart'' contracts from
the government even when controlling for lobbying, negotiation power,
and the employment of former government employees (Ferris, Houston, and
Javakhadze 2019).

Potentially, the influence exerted by contributors when making a
political contribution is so indirect that it doesn't always materialize
in statistical patterns of legislative voting, but there is evidence of
the influence as a result of the legislation. Interest groups have been
found to seek direct \emph{and} indirect access to the policy making
process (Fouirnaies 2018). One study found that firms that donated to
winning candidates experienced an abnormal equity return of 3\% compared
to firms that contributed to losing campaigns (Akey 2015). Another study
found that future returns of firms is positively and significantly
correlated with contributions in support for candidates, finding the
strongest effect among firms that support candidates within the state
that the firm is based (Cooper, Gulen, and Ovtchinnikov 2010). In
addition to immediately-felt financial returns, donors may
systematically contribute money to legislative agenda setters, such as
chairs of financial committees, in an effort to set future legislative
agendas (Fouirnaies 2018). Even campaign contribution from business
executives are ``best understood as purchases of `good will' whose
returns, while positive in expectation, are contingent and rare''
(Gordon, Hafer, and Landa 2007). Political contributions have also been
found to reduce the punishment for business executives who are
sanctioned for committing fraud (Fulmer, Knill, and Yu 2017).

Instead of focusing on direct access or financial outcomes, this
research article examines politicians' public support of policy issues.
Under the access-oriented/ influence model of political donor
motivations, we would expect to find politicians to be more supportive
of certain policy issues after receiving campaign contributions from
access-oriented donors, which leads to \(H_{1}\).

\textbf{\(H_{1}\): Donations from various coalitions of political donors
will precede, or Granger cause, increased public support of certain
political issues from the politicians to whom they donate.}

Since access-oriented donors are thought to be wealthier contributors,
sometimes seeking access for financial gain, this paper will also
examine the amount contributed by members of donor coalitions that are
accepted by \(H_{1}\). \(H_{2}\) also serves as a measure of face
validity for the theory of access-oriented motivation and this study's
measurement.

\textbf{\(H_{2}\): Donors from access-oriented coalitions will on
average be \emph{larger} contributors to political campaigns than donors
not in access-oriented coalitions.}

\hypertarget{consumption-model}{%
\section{Consumption Model}\label{consumption-model}}

While the access-oriented model is centered on donors \emph{influencing}
the political process, the consumption model is about donors
\emph{participating} in the political process. As Ansolabehere, de
Figueiredo, and Snyder (2003) concluded, ``In our view, campaign
contributing should not be viewed as an investment, but rather as a form
of consumption--or, in the language of politics, participation.'' They
put forward that individual donors are intrinsically motivated by
ideology. People don't receive a direct benefit from making a political
donation, but they do experience the indirect benefits of participating
in a political campaign that matches their ideology and excites them.
Said another way, for consumption motivated donors, making a
contribution is just an extension of voting on a participatory spectrum.
Under the consumption model of political donor motivations, donations
are a way for individuals to participate and be responsive to their
``perception of the stakes in the election'' (Hill and Huber 2017).

Ideological proximity, or the spatial distance between the ideology of
candidates and donors, has been found to be an important component in
explaining a donor's decision to make a contribution (Ensley 2009). even
more so than specific issue positions (Barber, Canes-Wrone, and Thrower
2019). The similarity between a donor's policy preferences and a
senator's roll calls is a predictor of whether a donor makes a
contribution (Barber, Canes-Wrone, and Thrower 2017). But this
relationship between individual donors' opinions and senators' roll-call
voting may be a recent phenomenon and has not existed historically
(Canes-Wrone and Gibson 2019). Divergence of ideology among the
candidates for an office, such as a more extreme political opponent,
does not seem to impact donors' decision to make a contribution (Ensley
2009).

Out-of-state donors have been found to have policy motivations for
making a donation in an effort to acquire surrogate representation
(Baker 2020b), but this finding didn't hold for in-district donations.
One study found that PACs for organized labor unions reduced
contributions to members of the U.S. House of Representatives when they
supported the North American Free Trade Agreement (NAFTA) (Engel and
Jackson 1998), suggesting that labor PACs responded to perceived changes
in ideological proximity opposed to doubling-down on their efforts to
potentially influence legislators who have become estranged from the
PAC's priorities. Labor unions have been found to participate in
``punishing'' legislators for their votes (Jansa and Hoyman 2018).
However, this punishment has been found to coax incumbents into changing
their position back to being pro-labor (Jansa 2019), suggesting that
there might actually be some influence-buying. However, recent studies
have found PACs in the oil and gas industry to be motivated by the
consumption model, opposed to access-oriented/ influence model, when
making contributions to candidates (Goldberg et al. 2020). Contributions
from cotton farmers or cotton PACs were also not found to be significant
predictors of votes to amend the 2008 Farm Bill (Callahan 2019).

All together, under the consumption model of donor motivations we would
expect public support of policy issues to attract political donors who
care about that policy, which leads to \(H_{3}\).

\textbf{\(H_{3}\): Public support from politicians on certain political
issues will precede, or Granger cause, donations from various coalitions
of political donors.}

Individual donors, as opposed to PACs, continue to make up a clear
majority of donations to political candidates (Heerwig 2016). And these
individual donors have been found to most often exhibit behavior
consistent with the consumption model of donations (Barber 2016; Heerwig
2016). Further, individual donors arguably play an even more central
role in politics more recently with the growth in small-dollar
individual donors.

With the rise of small-dollar donors on the internet and the assumption
that these small-dollar donors are motivated by the consumption model of
donor motivations, this paper will examine the amount of money
contributed by members of donor coalitions that are accepted by
\(H_{3}\). \(H_{4}\) also serves as a measure of face validity for the
theory of the consumption motivation model of political donors and this
study's measurement.

\textbf{\(H_{4}\): Donors from consumption coalitions will on average be
\emph{smaller} contributors to political campaigns than donors not in
consumption coalitions.}

\hypertarget{rise-of-small-dollar-donors}{%
\section{Rise of Small-Dollar
Donors}\label{rise-of-small-dollar-donors}}

The growing number of small-dollar donors in the political process
suggests that there will be more consumption-oriented donors in the
future. The anecdotal examples of the Bernie Sanders and Donald Trump
presidential campaigns, both of which received a large number of
small-dollar donors (Choma and Voght 2020), illustrate the
consumption-oriented model's connection to small-dollar donors. Small
dollar donors likely did not directly access or influence the politics
of the Sanders or Trump campaigns. Instead, donors reacted to their
messages and decided to move further down the participatory spectrum in
those campaigns. Individual contributors are mostly all participants in
politics without a an ulterior motive besides wanting to support the
campaign they are contributing to. Individual donors are ``fickle
financiers of elections'' whose donation habits can be distrupted by
little changes to their worlds such moving to an area that is more or
less Democratic or Republican (Kettler and Lyons 2019).

The Democratic Party as a whole has recently grown its proportion of
money that is coming from small-dollar donors (Albert and Raja 2020).
Incumbents have been able to sustain their small-dollar fundraising
programs (Heberlig and Larson 2020)--suggesting that this trend is not
going to go away. This growth in small-dollar donors has created a
donorate that is more demographically representative of America but is
more ideologically extreme (Albert and Raja 2020) and give
indiscriminately to incumbents, challengers, and open seat candidates
(Culberson, McDonald, and Robbins 2019). It is conceivable that
campaigns that rely on small donors will adopt rhetoric and tout their
``outsider'' status in an effort to activate these small, more
ideologically extreme donors (Arbour 2020). In the past, extremist
legislators have been found to be able to leverage politically divisive
and contentious moments into increased fundraising (Oklobdzija 2017). As
a result, some have predicted that small-dollar donors will polarize the
nation's politics even further (Oklobdzija 2017). Although legislators
who receive a large number of small-dollar donors aren't more polarized
in their voting in the next legislative session, legislators taking up a
more polarized agenda does increase the number of small-dollar donors
they attract in the subsequent election (Keena and Knight-Finley 2019),
providing further evidence for the consumption model of political donor
motivations. Other studies have agreed that mass donors are the cause of
partisan polarization (Raja and Wiltse 2012) and some do not (Harden and
Kirkland 2016). And so, even though small-dollar donors themselves may
not be polarizing, they may provide incentive for politicians to take
more polarized positions.

The rise in small-dollar donors has been driven primarily by
technological advancements (Albert and Raja 2020) including growing
sophistication with big data analytics (Walker and Nowlin 2018),
particularly in modeling political behaviors of individuals (Nickerson
and Rogers 2014). Digital firms, including Facebook, Twitter, and Google
embed themselves into political campaigns and serve as ``quasi-digital
consultants'' to the campaigns that shape the ``digital strategy,
content, and execution'' of campaigns (Kreiss and McGregor 2018). Along
with virtually every other component of political campaigns,
fundraising, especially from small-dollar donors, is moving online
(Chester and Montgomery 2017). While scholars remain skeptical of the
power of data analytics on political campaigns, firms have successfully
cultivated their images and businesses around the role of advanced data
methods on political campaigns (Simon 2019).

\hypertarget{online-fundraising}{%
\section{Online Fundraising}\label{online-fundraising}}

The study of the connection between the internet and financial donors
has historically been researched in the field of non-profit
organizational studies (Hazard 2003; DSW 2000; Miller 2009; Raihani and
Smith 2015) and not political science. The few studies that have
researched the connection between social media posts and political
fundraising have found a connection between the two (Wang et al. 2020).
Before political scientists studied the digital world and donations to
campaigns, the internet was seen more broadly as an agora public
discussion (anduiza2010; Gennaro and Dutton 2006; Zúñiga, Puig-I-Abril,
and Rojas 2009; Valenzuela, Kim, and Gil de Zúñiga 2011; Vesnic-Alujevic
2012), a hub of political organizing (Cogburn and Espinoza-Vasquez 2011;
Jost et al. 2018; Levenshus 2010), and a useful predictor of offline
political capital (Gil de Zúñiga, Jung, and Valenzuela 2012; PhD 2005).

Although the study of the internet as a medium is relatively new to
political science, research suggests that its communication methods are
similar to traditional political communication and can be extrapolated
to offline characteristics. The differences that are seen in online
political communication, like lowered costs and eased barriers to entry,
represent a ``difference-of-degree'' and not a paradigm shifting
``difference-in-kind'' (Karpf 2010). There is a strong connection
between online channels of communication in the form of social networks
and offline connections and building and maintaining social capital from
those offline connections (Cranshaw et al. 2010; Ellison and Steinfield
2006; Liben-Nowell et al. 2005; Scellato et al. 2010). Online social
networks have also been used to study offline-based actions and beliefs
like opinion polarization (Lee et al. 2014), political polarization
(Hanna et al. 2013), political participation (Lawrence, Sides, and
Farrell 2010) and political discourse (Kushin and Kitchener 2009).

The bottom line is that online actions and behaviors have been found to
reflect the offline world, and the online world is frequently
extrapolated to explain offline actions and behaviors by prior
researchers. This study builds upon these previous uses of online
indicators of offline actions and beliefs by combining political
administrative records of political donations and politicians' social
media accounts to discern the relationship between political donations
and public support of policy issues.

Using social media will allow this paper to analyze the textual and
linguistic characteristics of the posts. Previous research has been able
to study the connection between digital language and political behaviors
such as protests {[}needcite{]} but not donations. This paper will be
able to use the connections found in \(H_{1}\) and \(H_{2}\) to explore
whether political sophistication (Benoit, Munger, and Spirling 2019),
polarization (Goet 2019; Lauderdale and Herzog 2016) or other textual
features of the posts that are connected to different models of donor
motivation are unique compared to posts on other topics that are not
found to have a relationship with coalitions of political donors.

\textbf{\(R_{1}\): Do social media posts from politicians that have been
found to either Granger cause or be Granger caused by donations from
coalitions of donors have unique textual characteristics?}

\hypertarget{data}{%
\section{Data}\label{data}}

Data for this research comes from two primary sources: politicians'
social media posts and political donation data. For social media posts,
this paper used the Facebook (Barbera, Geisler, and Atteveldt 2017) and
Twitter (Kearney 2019) APIs to collect social media posts from all
candidates for the Wisconsin State Senate and Wisconsin State Assembly
during the 2016 election cycle (\emph{n} = 82,851). A subset of these
posts were hand-coded into 27 topical categories. This subset was used
to train a BERT deep learning transfer model that was used to predict
the topic of the remainder of the posts (training dataset = 8,242, 10\%
of total posts; testing dataset = 4,122, 5\% of total posts). Political
donation data for all candidates to the Wisconsin State Legislature
during the 2016 election cycle were collected from the Wisconsin
Campaign Information System (CFIS) (\emph{n} = 12,962). These donations
were used to create a network of political donations with candidates and
donors serving as nodes and donations between them as edges. This
network was clustered into distinct communities so that donors in each
community are most similar to one another based on which campaigns they
contributed to. I theorize that these clusters of donors represent
\emph{latent coalitions} of donors who, whether they operate in an
organized fashion or not, are working toward the goal of electing the
same candidates.

\hypertarget{methodology}{%
\section{Methodology}\label{methodology}}

These two datasets were analyzed against each other using the Granger
causality time-series methodology. This methodology has been used by
other researchers to study social media (Freelon, McIlwain, and Clark
2018; Lukito 2020). Similar to political donations, this methodology has
been used to study the relationship between social media and non-social
media events such as offline protests (Bastos, Mercea, and Charpentier
2015) and stock prices (Park, Leung, and Ma 2017). Granger causality
detects whether movements in one time series precedes, lags, has a
confounding variable, or is not related to another time series.
Specifically, this paper compares time series of donations from clusters
of political donors and time series of the number of social media posts
by each topic that were made by campaigns that each donor cluster
contributed to. For example, a time series of donations from a donor
coalition was compared to the aggregate count of posts about a given
topic made by candidates that the donor cluster contributed to.

\hypertarget{preliminary-results}{%
\section{Preliminary Results}\label{preliminary-results}}

Initial results suggest that it is more common to observe behavior
consistent with the consumption model (31\% of coalitions, 4/13) than
the access-oriented model. However, the access-oriented model is still
observed in 15\% of coalitions (2/13). Under a strict interpretation of
either model, we would expect to find behavior that fits only with that
model. These results that find both the models present in the data is in
line with some other research in suggesting that there are a ``diversity
of roles individual contributors play in the campaign finance system''
(Heerwig 2016). Specific results of the Granger causality model are in
Figure 1 below.

\begin{figure}
\centering
\includegraphics{../tables_and_figures/aejmc_abstract_1.jpg}
\caption{Donor Motivation Models}
\end{figure}

One theoretical next step for this paper is to flesh out the
implications of observing behavior that fits under both the consumption
and access-oriented model of political donors. Most often, the
literature assumes that political donors have monolithic a monolithic
psychological process that motivate them. However, the clear breakdown
of different coalitions exhibiting behavior that falls into different
models, and distinct behavior in relation to unique policy issues,
suggests that latent coalitions of political donors are strategic actors
with unique motivations. One empirical next step is to quantify
potential confounders for donor clusters that don't fit under either
model, such as geographic proximity or competitiveness of the races
contributed to.

\hypertarget{references}{%
\section*{References}\label{references}}
\addcontentsline{toc}{section}{References}

\hypertarget{refs}{}
\leavevmode\hypertarget{ref-akey2015}{}%
Akey, Pat. 2015. ``Valuing Changes in Political Networks: Evidence from
Campaign Contributions to Close Congressional Elections.'' \emph{The
Review of Financial Studies} 28 (11): 3188--3223.
\url{https://doi.org/10.1093/rfs/hhv035}.

\leavevmode\hypertarget{ref-albert2020}{}%
Albert, Zachary, and Raymond La Raja. 2020. ``Small Dollar Donors and
the Evolving Democratic Party.'' \emph{APSA Preprints}.

\leavevmode\hypertarget{ref-ansolabehere2003}{}%
Ansolabehere, Stephen, John M. de Figueiredo, and James M. Snyder. 2003.
``Why Is There so Little Money in U.s. Politics.'' \emph{Journal of
Economic Perspectives} 17 (1): 105--30.

\leavevmode\hypertarget{ref-arbour2020}{}%
Arbour, Brian. 2020. ``Tiny Donations, Big Impact: How Small-Dollar
Donors Are Eroding the Power of Party Insiders.'' \emph{Society} 57:
496--506.

\leavevmode\hypertarget{ref-austensmith1995}{}%
Austen-Smith, David. 1995. ``Campaign Contributions and Access.''
\emph{The American Political Science Review} 89 (3): 566--81.
\url{http://www.jstor.org/stable/2082974}.

\leavevmode\hypertarget{ref-baker2020a}{}%
Baker, Anne. 2020a. ``Policies, Profits, Networks, or Duty?: Donors'
Motivations for Contributing to Parties and Interest Groups.'' \emph{The
Social Science Journal} 0 (0): 1--16.
\url{https://doi.org/10.1080/03623319.2020.1727224}.

\leavevmode\hypertarget{ref-baker2020b}{}%
---------. 2020b. ``The Partisan and Policy Motivations of Political
Donors Seeking Surrogate Representation in House Elections.''
\emph{Political Behavior} 42 (4): 1035--54.
\url{https://doi.org/10.1007/s11109-019-09531-2}.

\leavevmode\hypertarget{ref-barber2016a}{}%
Barber, Michael. 2016. ``Donation Motivations: Testing Theories of
Access and Ideology.'' \emph{Political Research Quarterly} 69 (1):
148--59.

\leavevmode\hypertarget{ref-barber2019}{}%
Barber, Michael, Brandice Canes-Wrone, and Sharece Thrower. 2019.
``Campaign Contributions and Donors' Policy Agreement with Presidential
Candidates.'' \emph{Presidential Studies Quarterly} 49 (4): 770--97.
\url{https://doi.org/https://doi.org/10.1111/psq.12609}.

\leavevmode\hypertarget{ref-barber2017}{}%
Barber, Michael J., Brandice Canes-Wrone, and Sharece Thrower. 2017.
``Ideologically Sophisticated Donors: Which Candidates Do Individual
Contributors Finance?'' \emph{American Journal of Political Science} 61
(2): 271--88. \url{https://doi.org/https://doi.org/10.1111/ajps.12275}.

\leavevmode\hypertarget{ref-rfacebook}{}%
Barbera, Pablo, Andrew Geisler, and Wouter van Atteveldt. 2017.
\emph{Rfacebook}.
\url{https://cran.r-project.org/web/packages/Rfacebook/Rfacebook.pdf}.

\leavevmode\hypertarget{ref-bastos2015}{}%
Bastos, Marco T., Dan Mercea, and Arthur Charpentier. 2015. ``Tents,
Tweets, and Events: The Interplay Between Ongoing Protests and Social
Media.'' \emph{Journal of Communication} 65 (2): 320--50.
\url{https://doi.org/10.1111/jcom.12145}.

\leavevmode\hypertarget{ref-benoit2019}{}%
Benoit, Kenneth, Kevin Munger, and Arthur Spirling. 2019. ``Measuring
and Explaining Political Sophistication Through Textual Complexity.''
\emph{American Journal of Political Science} 63 (2): 491--508.
\url{https://doi.org/https://doi.org/10.1111/ajps.12423}.

\leavevmode\hypertarget{ref-bonica2016}{}%
Bonica, Adam. 2016. ``Avenues of Influence: On the Political
Expenditures of Corporations and Their Directors and Executives.''
\emph{Business and Politics} 18 (4): 367--94.
\url{https://doi.org/10.1515/bap-2016-0004}.

\leavevmode\hypertarget{ref-butler2011}{}%
Butler, Daniel M., and David W. Nickerson. 2011. ``Can Learning
Constituency Opinion Affect How Legislators Vote? Results from a Field
Experiment.'' \emph{Quarterly Journal of Political Science} 6: 55--83.

\leavevmode\hypertarget{ref-callahan2019}{}%
Callahan, Scott. 2019. ``Do Campaign Contributions from Farmers
Influence Agricultural Policy? Evidence from a 2008 Farm Bill Amendment
Vote to Curtail Cotton Subsidies.'' \emph{Journal of Agricultural and
Applied Economics} 51 (3): 417--33.
\url{https://doi.org/10.1017/aae.2019.9}.

\leavevmode\hypertarget{ref-caneswrone2019}{}%
Canes-Wrone, Brandice, and Nathan Gibson. 2019. ``Does Money Buy
Congressional Love? Individual Donors and Legislative Voting.''
\emph{Congress \& the Presidency} 46 (1): 1--27.
\url{https://doi.org/10.1080/07343469.2018.1518965}.

\leavevmode\hypertarget{ref-chester2017}{}%
Chester, Jeff, and Kathryn C. Montgomery. 2017. ``The Role of Digital
Marketing in Political Campaigns.'' \emph{Internet Policy Review} 6 (4):
1--20. \url{https://doi.org/10.14763/2017.4.773}.

\leavevmode\hypertarget{ref-choma2020}{}%
Choma, Russ, and Kara Voght. 2020. ``Small-Dollar Donors Powered the
2020 Race. Then the Pandemic Happened.'' \emph{Mother Jones}, April.

\leavevmode\hypertarget{ref-cogburn2011}{}%
Cogburn, Derrick L., and Fatima K. Espinoza-Vasquez. 2011. ``From
Networked Nominee to Networked Nation: Examining the Impact of Web 2.0
and Social Media on Political Participation and Civic Engagement in the
2008 Obama Campaign.'' \emph{Journal of Political Marketing} 10 (1-2):
189--213. \url{https://doi.org/10.1080/15377857.2011.540224}.

\leavevmode\hypertarget{ref-constant2006}{}%
Constant, Louay M. 2006. ``When Money Matters: Campaign Contributions,
Roll Call Votes, and School Choice in Florida.'' \emph{State Politics \&
Policy Quarterly} 6 (2): 195--219.
\url{https://doi.org/10.1177/153244000600600204}.

\leavevmode\hypertarget{ref-cooper2010}{}%
Cooper, Michael J., Huseyin Gulen, and Alexei V. Ovtchinnikov. 2010.
``Corporate Political Contributions and Stock Returns.'' \emph{The
Journal of Finance} 65 (2): 687--724.
\url{https://doi.org/https://doi.org/10.1111/j.1540-6261.2009.01548.x}.

\leavevmode\hypertarget{ref-cranshaw2010}{}%
Cranshaw, Justin, Eran Toch, J. Hong, A. Kittur, and N. Sadeh. 2010.
``Bridging the Gap Between Physical Location and Online Social
Networks.'' \emph{Proceedings of the 12th ACM International Conference
on Ubiquitous Computing}.

\leavevmode\hypertarget{ref-culberson2019}{}%
Culberson, Tyler, Michael P. McDonald, and Suzanne M. Robbins. 2019.
``Small Donors in Congressional Elections.'' \emph{American Politics
Research} 47 (5): 970--99.
\url{https://doi.org/10.1177/1532673X18763918}.

\leavevmode\hypertarget{ref-marx2000}{}%
DSW, Jerry D. Marx. 2000. ``Online Fundraising in the Human Services.''
\emph{Journal of Technology in Human Services} 17 (2-3): 137--52.
\url{https://doi.org/10.1300/J017v17n02/_03}.

\leavevmode\hypertarget{ref-edwards2016}{}%
Edwards, Geoff, and Rui de Figueiredo. 2016. ``The Market for
Legislative Influence over Regulatory Policy'' 34 (May).
\url{https://doi.org/10.1108/S0742-332220160000034007}.

\leavevmode\hypertarget{ref-ellison2006}{}%
Ellison, Nicole, and Charles Steinfield. 2006. ``Spatially Bounded
Online Social Networks and Social Capital: The Role of Facebook.''
\emph{Annual Conference of the International Communication Association},
January.

\leavevmode\hypertarget{ref-engel1998}{}%
Engel, Steven T., and David J. Jackson. 1998. ``Wielding the Stick
Instead of the Carrot: Labor Pac Punishment of Pro-Nafta Democrats.''
\emph{Political Research Quarterly} 51 (3): 813--28.
\url{https://doi.org/10.1177/106591299805100312}.

\leavevmode\hypertarget{ref-ensley2009}{}%
Ensley, Michael J. 2009. ``Individual Campaign Contributions and
Candidate Ideology.'' \emph{Public Choice} 138 (1/2): 221--38.
\url{http://www.jstor.org/stable/40270840}.

\leavevmode\hypertarget{ref-fellowes2004}{}%
Fellowes, Matthew C., and Patrick J. Wolf. 2004. ``Funding Mechanisms
and Policy Instruments: How Business Campaign Contributions Influence
Congressional Votes.'' \emph{Political Research Quarterly} 57 (2):
315--24.

\leavevmode\hypertarget{ref-ferris2019}{}%
Ferris, Stephen P., Reza Houston, and David Javakhadze. 2019. ``It Is a
Sweetheart of a Deal: Political Connections and Corporate-Federal
Contracting.'' \emph{Financial Review} 54 (1): 57--84.
\url{https://doi.org/https://doi.org/10.1111/fire.12181}.

\leavevmode\hypertarget{ref-finger2019}{}%
Finger, Leslie K. 2019. ``Interest Group Influence and the Two Faces of
Power.'' \emph{American Politics Research} 47 (4): 852--86.
\url{https://doi.org/10.1177/1532673X18786723}.

\leavevmode\hypertarget{ref-fouirnaies2018}{}%
Fouirnaies, Alexander. 2018. ``When Are Agenda Setters Valuable?''
\emph{American Journal of Political Science} 62 (1): 176--91.
\url{https://doi.org/https://doi.org/10.1111/ajps.12316}.

\leavevmode\hypertarget{ref-fouirnaies2015}{}%
Fouirnaies, Alexander, and Andrew Hall. 2015. ``The Exposure Theory of
Access: Why Some Firms Seek More Access to Incumbents Than Others.''
\emph{SSRN Electronic Journal}, January.
\url{https://doi.org/10.2139/ssrn.2652361}.

\leavevmode\hypertarget{ref-francia2003}{}%
Francia, Peter L., John C. Green, Paul S. Herrnson, Lynda W. Powell, and
and Clyde Wilcox. 2003. \emph{The Financiers of Congressional
Elections}. New York, NY: Columbia University Press.

\leavevmode\hypertarget{ref-freelon2018}{}%
Freelon, D, C McIlwain, and M Clark. 2018. ``Quantifying the Power and
Consequences of Social Media Protest.'' \emph{New Media \& Society} 20
(3): 990--1011. \url{https://doi.org/10.1177/1461444816676646}.

\leavevmode\hypertarget{ref-fremeth2013}{}%
Fremeth, Adam, Brian Kelleher Richter, and Brandon Schaufele. 2013.
``Campaign Contributions over Ceos' Careers.'' \emph{American Economic
Journal: Applied Economics} 5 (3): 170--88.
\url{https://doi.org/10.1257/app.5.3.170}.

\leavevmode\hypertarget{ref-fu2020}{}%
Fu, Shu, and William G. Howell. 2020. ``The Behavioral Consequences of
Public Appeals: Evidence on Campaign Fundraising from the 2018
Congressional Elections.'' \emph{Presidential Studies Quarterly} 50 (2):
325--47. \url{https://doi.org/https://doi.org/10.1111/psq.12645}.

\leavevmode\hypertarget{ref-fulmer2017}{}%
Fulmer, Sarah, A. Knill, and X. Yu. 2017. ``Negation of Sanctions: The
Personal Effect of Political Contributions.'' \emph{Business History
eJournal}.

\leavevmode\hypertarget{ref-degennaro2006}{}%
Gennaro, Corinna di, and William Dutton. 2006. ``The Internet and the
Public: Online and Offline Political Participation in the United
Kingdom.'' \emph{Parliamentary Affairs} 59 (2): 299--313.
\url{https://doi.org/10.1093/pa/gsl004}.

\leavevmode\hypertarget{ref-zuniga2012}{}%
Gil de Zúñiga, Homero, Nakwon Jung, and Sebastián Valenzuela. 2012.
``Social Media Use for News and Individuals' Social Capital, Civic
Engagement and Political Participation.'' \emph{Journal of
Computer-Mediated Communication} 17 (3): 319--36.
\url{https://doi.org/10.1111/j.1083-6101.2012.01574.x}.

\leavevmode\hypertarget{ref-goet2019}{}%
Goet, Niels D. 2019. ``Measuring Polarization with Text Analysis:
Evidence from the Uk House of Commons, 1811--2015.'' \emph{Political
Analysis} 27 (4): 518--39. \url{https://doi.org/10.1017/pan.2019.2}.

\leavevmode\hypertarget{ref-goldberg2020}{}%
Goldberg, Matthew H., Jennifer R. Marlon, Xinran Wang, Sander van der
Linden, and Anthony Leiserowitz. 2020. ``Oil and Gas Companies Invest in
Legislators That Vote Against the Environment.'' \emph{Proceedings of
the National Academy of Sciences} 117 (10): 5111--2.
\url{https://doi.org/10.1073/pnas.1922175117}.

\leavevmode\hypertarget{ref-gordon2007}{}%
Gordon, Sanford C., Catherine Hafer, and Dimitri Landa. 2007.
``Consumption or Investment? On Motivations for Political Giving.''
\emph{The Journal of Politics} 69 (4): 1057--72.

\leavevmode\hypertarget{ref-gordon2001}{}%
Gordon, Stacy B. 2001. ``All Votes Are Not Created Equal: Campaign
Contributions and Critical Votes.'' \emph{The Journal of Politics} 63
(1): 249--69. \url{https://doi.org/10.1111/0022-3816.00067}.

\leavevmode\hypertarget{ref-grenzke1989}{}%
Grenzke, Janet M. 1989. ``PACs and the Congressional Supermarket: The
Currency Is Complex.'' \emph{American Journal of Political Science} 33
(1): 1--24. \url{http://www.jstor.org/stable/2111251}.

\leavevmode\hypertarget{ref-hadani2017}{}%
Hadani, Michael, Jean-Philippe Bonardi, and Nicolas M Dahan. 2017.
``Corporate Political Activity, Public Policy Uncertainty, and Firm
Outcomes: A Meta-Analysis.'' \emph{Strategic Organization} 15 (3):
338--66. \url{https://doi.org/10.1177/1476127016651001}.

\leavevmode\hypertarget{ref-hall1990}{}%
Hall, Richard L., and Frank W. Wayman. 1990. ``Buying Time: Moneyed
Interests and the Mobilization of Bias in Congressional Committees.''
\emph{American Political Science Review} 84 (3): 797--820.
\url{https://doi.org/10.2307/1962767}.

\leavevmode\hypertarget{ref-hanna2013}{}%
Hanna, Alex, Chris Wells, Peter Maurer, Lew Friedland, Dhavan Shah, and
Jörg Matthes. 2013. ``Partisan Alignments and Political Polarization
Online: A Computational Approach to Understanding the French and Us
Presidential Elections.'' In \emph{Proceedings of the 2nd Workshop on
Politics, Elections and Data}, 15--22. PLEAD '13. New York, NY, USA:
Association for Computing Machinery.
\url{https://doi.org/10.1145/2508436.2508438}.

\leavevmode\hypertarget{ref-harden2016}{}%
Harden, Jeffrey J., and Justin H. Kirkland. 2016. ``Do Campaign Donors
Influence Polarization? Evidence from Public Financing in the American
States.'' \emph{Legislative Studies Quarterly} 41 (1): 119--52.
\url{https://doi.org/https://doi.org/10.1111/lsq.12108}.

\leavevmode\hypertarget{ref-hayes2017}{}%
Hayes, Thomas J. 2017. ``Bankruptcy Reform and Congressional Action: The
Role of Organized Interests in Shaping Policy.'' \emph{Social Science
Research} 64: 67--78.
\url{https://doi.org/https://doi.org/10.1016/j.ssresearch.2016.09.026}.

\leavevmode\hypertarget{ref-hazard2003}{}%
Hazard, Brenda L. 2003. ``Online Fundraising at Arl Libraries.''
\emph{The Journal of Academic Librarianship} 29 (1): 8--15.
\url{https://doi.org/https://doi.org/10.1016/S0099-1333(02)00399-3}.

\leavevmode\hypertarget{ref-heberlig2020}{}%
Heberlig, Eric, and Bruce Larson. 2020. ``Gender and Small
Contributions: Fundraising by the Democratic Freshman Class of 2018 in
the 2020 Election.'' \emph{Society} 57: 534--39.
\url{https://doi.org/https://doi.org/10.1007/s12115-020-00528-w}.

\leavevmode\hypertarget{ref-heerwig2016}{}%
Heerwig, Jennifer A. 2016. ``Donations and Dependence: Individual
Contributor Strategies in House Elections.'' \emph{Social Science
Research} 60: 181--98.
\url{https://doi.org/https://doi.org/10.1016/j.ssresearch.2016.06.001}.

\leavevmode\hypertarget{ref-herndon1982}{}%
Herndon, James F. 1982. ``Access, Record, and Competition as Influences
on Interest Group Contributions to Congressional Campaigns.'' \emph{The
Journal of Politics} 44 (4): 996--1019.

\leavevmode\hypertarget{ref-hill2017}{}%
Hill, Seth J., and Gregory A. Huber. 2017. ``Representativeness and
Motivations of the Contemporary Donorate: Results from Merged Survey and
Administrative Records.'' \emph{Political Behavior} 39 (1): 3--29.
\url{https://doi.org/10.1007/s11109-016-9343-y}.

\leavevmode\hypertarget{ref-hogan2020}{}%
Hogan, Robert E. 2020. ``Legislative Voting and Environmental
Policymaking in the American States.'' \emph{Environmental Politics} 0
(0): 1--20. \url{https://doi.org/10.1080/09644016.2020.1788897}.

\leavevmode\hypertarget{ref-jansa2019}{}%
Jansa, Joshua M. 2019. ``You Catch More Flies with Honey: An Analysis of
Pac Punishment and Congressional Vote Switching.'' \emph{Interest Groups
\& Advocacy} 8 (2).

\leavevmode\hypertarget{ref-jansa2018}{}%
Jansa, Joshua M., and Michele M. Hoyman. 2018. ``Do Unions Punish
Democrats? Free-Trade Votes and Labor Pac Contributions, 1999--2012.''
\emph{Political Research Quarterly} 71 (2): 424--39.
\url{https://doi.org/10.1177/1065912917738575}.

\leavevmode\hypertarget{ref-jost2018}{}%
Jost, John, Pablo Barberá, Richard Bonneau, Melanie Langer, Megan
Metzger, Jonathan Nagler, Joanna Sterling, and Joshua Tucker. 2018.
``How Social Media Facilitates Political Protest: Information,
Motivation, and Social Networks: Social Media and Political Protest.''
\emph{Political Psychology} 39 (February): 85--118.
\url{https://doi.org/10.1111/pops.12478}.

\leavevmode\hypertarget{ref-kalla2015}{}%
Kalla, Joshua L., and David E. Broockman. 2016. ``Campaign Contributions
Facilitate Access to Congressional Officials: A Randomized Field
Experiment.'' \emph{American Journal of Political Science} 60 (3):
545--58. \url{https://doi.org/https://doi.org/10.1111/ajps.12180}.

\leavevmode\hypertarget{ref-karpf2010}{}%
Karpf, David. 2010. ``Online Political Mobilization from the Advocacy
Group's Perspective: Looking Beyond Clicktivism.'' \emph{Policy \&
Internet} 2 (4): 7--41.
\url{https://doi.org/https://doi.org/10.2202/1944-2866.1098}.

\leavevmode\hypertarget{ref-rtweet}{}%
Kearney, Michael W. 2019. ``Rtweet: Collecting and Analyzing Twitter
Data.'' \emph{Journal of Open Source Software} 4 (42): 1829.
\url{https://doi.org/10.21105/joss.01829}.

\leavevmode\hypertarget{ref-keena2019}{}%
Keena, Alex, and Misty Knight-Finley. 2019. ``Are Small Donors
Polarizing? A Longitudinal Study of the Senate.'' \emph{Election Law
Journal: Rules, Politics, and Policy} 18 (2): 132--44.
\url{https://doi.org/10.1089/elj.2018.0498}.

\leavevmode\hypertarget{ref-kettler2019}{}%
Kettler, Jaclyn J., and Jeffrey Lyons. 2019. ``The Fickle Financiers of
Elections? The Impact of Moving on Individual Contributions.''
\emph{Journal of Elections, Public Opinion and Parties} 0 (0): 1--19.
\url{https://doi.org/10.1080/17457289.2019.1652620}.

\leavevmode\hypertarget{ref-kreiss2018}{}%
Kreiss, Daniel, and Shannon C. McGregor. 2018. ``Technology Firms Shape
Political Communication: The Work of Microsoft, Facebook, Twitter, and
Google with Campaigns During the 2016 U.s. Presidential Cycle.''
\emph{Political Communication} 35 (2): 155--77.
\url{https://doi.org/10.1080/10584609.2017.1364814}.

\leavevmode\hypertarget{ref-kushin2009}{}%
Kushin, Matthew J., and Kelin Kitchener. 2009. ``Getting Political on
Social Network Sites: Exploring Online Political Discourse on
Facebook.'' \emph{First Monday} 14 (11).
\url{https://doi.org/10.5210/fm.v14i11.2645}.

\leavevmode\hypertarget{ref-langbein1986}{}%
Langbein, Laura I. 1986. ``Money and Access: Some Empirical Evidence.''
\emph{The Journal of Politics} 40 (4): 1052--62.

\leavevmode\hypertarget{ref-lauderdale2016}{}%
Lauderdale, Benjamin E., and Alexander Herzog. 2016. ``Measuring
Political Positions from Legislative Speech.'' \emph{Political Analysis}
24 (3): 374--94. \url{https://doi.org/10.1093/pan/mpw017}.

\leavevmode\hypertarget{ref-lawrence2010}{}%
Lawrence, Eric, John Sides, and Henry Farrell. 2010. ``Self-Segregation
or Deliberation? Blog Readership, Participation, and Polarization in
American Politics.'' \emph{Perspectives on Politics} 8 (1): 141--57.
\url{http://www.jstor.org/stable/25698520}.

\leavevmode\hypertarget{ref-lee2014}{}%
Lee, Jae Kook, Jihyang Choi, Cheonsoo Kim, and Yonghwan Kim. 2014.
``Social Media, Network Heterogeneity, and Opinion Polarization.''
\emph{Journal of Communication} 64 (4): 702--22.
\url{https://doi.org/10.1111/jcom.12077}.

\leavevmode\hypertarget{ref-levenshus2010}{}%
Levenshus, Abbey. 2010. ``Online Relationship Management in a
Presidential Campaign: A Case Study of the Obama Campaign's Management
of Its Internet-Integrated Grassroots Effort.'' \emph{Journal of Public
Relations Research} 22 (3): 313--35.
\url{https://doi.org/10.1080/10627261003614419}.

\leavevmode\hypertarget{ref-liben2005}{}%
Liben-Nowell, David, Jasmine Novak, Ravi Kumar, Prabhakar Raghavan, and
Andrew Tomkins. 2005. ``Geographic Routing in Social Networks.''
\emph{Proceedings of the National Academy of Sciences} 102 (33):
11623--8. \url{https://doi.org/10.1073/pnas.0503018102}.

\leavevmode\hypertarget{ref-lukito2020}{}%
Lukito, Josephine. 2020. ``Coordinating a Multi-Platform Disinformation
Campaign: Internet Research Agency Activity on Three U.s. Social Media
Platforms, 2015 to 2017.'' \emph{Political Communication} 37 (2):
238--55. \url{https://doi.org/10.1080/10584609.2019.1661889}.

\leavevmode\hypertarget{ref-mckay2010}{}%
McKay, Amy. 2010. ``The Effects of Interest Groups' Ideology on Their
Pac and Lobbying Expenditures.'' \emph{Business and Politics} 12 (2):
1--21. \url{https://doi.org/10.2202/1469-3569.1306}.

\leavevmode\hypertarget{ref-mckay2018}{}%
---------. 2018. ``What Do Campaign Contributions Buy? Lobbyists'
Strategic Giving.'' \emph{Interest Groups \& Advocacy} 7 (1).

\leavevmode\hypertarget{ref-milbrath1958}{}%
Milbrath, Lester W. 1958. ``The Political Party Activity of Washington
Lobbyists.'' \emph{The Journal of Politics} 20 (2): 339--52.

\leavevmode\hypertarget{ref-miller2009}{}%
Miller, Bryan. 2009. ``Community Fundraising 2.0---the Future of
Fundraising in a Networked Society?'' \emph{International Journal of
Nonprofit and Voluntary Sector Marketing} 14 (4): 365--70.
\url{https://doi.org/https://doi.org/10.1002/nvsm.373}.

\leavevmode\hypertarget{ref-nickerson2014}{}%
Nickerson, David W., and Todd Rogers. 2014. ``Political Campaigns and
Big Data.'' \emph{Journal of Economic Perspectives} 28 (2): 51--74.
\url{https://doi.org/10.1257/jep.28.2.51}.

\leavevmode\hypertarget{ref-oklobdzija2017}{}%
Oklobdzija, Stan. 2017. ``Closing down and Cashing in: Extremism and
Political Fundraising.'' \emph{State Politics \& Policy Quarterly} 17
(2): 201--24. \url{https://doi.org/10.1177/1532440016679373}.

\leavevmode\hypertarget{ref-park2017}{}%
Park, J., H. Leung, and K. Ma. 2017. ``Information Fusion of Stock
Prices and Sentiment in Social Media Using Granger Causality.'' In
\emph{2017 Ieee International Conference on Multisensor Fusion and
Integration for Intelligent Systems (Mfi)}, 614--19.
\url{https://doi.org/10.1109/MFI.2017.8170390}.

\leavevmode\hypertarget{ref-hardina2005}{}%
PhD, Donna Hardina. 2005. ``Using the Web to Teach Power Analysis.''
\emph{The Social Policy Journal} 4 (2): 51--68.
\url{https://doi.org/10.1300/J185v04n02/_05}.

\leavevmode\hypertarget{ref-powell2016}{}%
Powell, Eleanor Neff, and Justin Grimmer. 2016. ``Money in Exile:
Campaign Contributions and Committee Access.'' \emph{The Journal of
Politics} 78 (4): 974--88. \url{https://doi.org/10.1086/686615}.

\leavevmode\hypertarget{ref-raihani2015}{}%
Raihani, Nichola J., and Sarah Smith. 2015. ``Competitive Helping in
Online Giving.'' \emph{Current Biology} 25 (9): 1183--6.
\url{https://doi.org/https://doi.org/10.1016/j.cub.2015.02.042}.

\leavevmode\hypertarget{ref-laraja2012}{}%
Raja, Raymond J. La, and David L. Wiltse. 2012. ``Don't Blame Donors for
Ideological Polarization of Political Parties: Ideological Change and
Stability Among Political Dontributors, 1972-2008.'' \emph{American
Politics Research} 40 (3): 501--30.
\url{https://doi.org/https://doi.org/10.1177/1532673X11429845}.

\leavevmode\hypertarget{ref-rhodes2018}{}%
Rhodes, Jesse H., Brian F. Schaffner, and Raymond J. La Raja. 2018.
``Detecting and Understanding Donor Strategies in Midterm Elections.''
\emph{Political Research Quarterly} 71 (3): 503--16.
\url{https://doi.org/10.1177/1065912917749323}.

\leavevmode\hypertarget{ref-roscoe2005}{}%
Roscoe, Douglas D., and Shannon Jenkins. 2005. ``A Meta-Analysis of
Campaign Contributions' Impact on Roll Call Voting*.'' \emph{Social
Science Quarterly} 86 (1): 52--68.
\url{https://doi.org/https://doi.org/10.1111/j.0038-4941.2005.00290.x}.

\leavevmode\hypertarget{ref-scellato2010}{}%
Scellato, Salvatore, Cecilia Mascolo, Mirco Musolesi, and Vito Latora.
2010. ``Distance Matters: Geo-Social Metrics for Online Social
Networks.'' In, 8. WOSN'10. USA: USENIX Association.

\leavevmode\hypertarget{ref-schnakenberg2021}{}%
Schnakenberg, Keith E., and Ian R. Turner. 2021. ``Helping Friends or
Influencing Foes: Electoral and Policy Effects of Campaign Finance
Contributions.'' \emph{American Journal of Political Science} 65 (1):
88--100. \url{https://doi.org/https://doi.org/10.1111/ajps.12534}.

\leavevmode\hypertarget{ref-simon2019}{}%
Simon, Felix M. 2019. ```We Power Democracy': Exploring the Promises of
the Political Data Analytics Industry.'' \emph{The Information Society}
35 (3): 158--69. \url{https://doi.org/10.1080/01972243.2019.1582570}.

\leavevmode\hypertarget{ref-valenzuela2011}{}%
Valenzuela, Sebastián, Yonghwan Kim, and Homero Gil de Zúñiga. 2011.
``Social Networks that Matter: Exploring the Role of Political
Discussion for Online Political Participation.'' \emph{International
Journal of Public Opinion Research} 24 (2): 163--84.
\url{https://doi.org/10.1093/ijpor/edr037}.

\leavevmode\hypertarget{ref-vesnic2012}{}%
Vesnic-Alujevic, Lucia. 2012. ``Political Participation and Web 2.0 in
Europe: A Case Study of Facebook.'' \emph{Public Relations Review} 38
(3): 466--70.
\url{https://doi.org/https://doi.org/10.1016/j.pubrev.2012.01.010}.

\leavevmode\hypertarget{ref-wahl2018}{}%
Wahl, S., and J. Sheppard. 2018. ``Association Rule Mining in Fuzzy
Political Donor Communities.'' In \emph{MLDM 2018: Machine Learning and
Data Mining in Pattern Recognition}. Vol. 10935.
\url{https://doi.org/https://doi.org/10.1007/978-3-319-96133-0_18}.

\leavevmode\hypertarget{ref-wahl2019}{}%
Wahl, S., J. Sheppard, and E. Shanahan. 2019. ``Legislative Vote
Prediction Using Campaign Donations and Fuzzy Hierarchical
Communities.'' In \emph{2019 18th Ieee International Conference on
Machine Learning and Applications (Icmla)}, 718--25.
\url{https://doi.org/10.1109/ICMLA.2019.00129}.

\leavevmode\hypertarget{ref-walker2018}{}%
Walker, Doug, and Edward L. Nowlin. 2018. ``Data-Driven Precision and
Selectiveness in Political Campaign Fundraising.'' \emph{Journal of
Political Marketing} 0 (0): 1--20.
\url{https://doi.org/10.1080/15377857.2018.1457590}.

\leavevmode\hypertarget{ref-wang2020}{}%
Wang, Austin Horng-En, Fei-Pei Lai, Fushun Hsu, and Peter Shaojui Wang.
2020. ``Mobilizing Sophisticated Donors: What Candidate Facebook Posts
Do Attract Intra- and Inter-District Donations?'' \emph{Issues \&
Studies} 56 (04): 2050005.
\url{https://doi.org/10.1142/S1013251120500058}.

\leavevmode\hypertarget{ref-wayman1985}{}%
Wayman, Frank Whelon. 1985. ``Arms Control and Strategic Arms Voting in
the U.s. Senate: Patterns of Change, 1967-1983.'' \emph{The Journal of
Conflict Resolution} 29 (2): 225--51.
\url{http://www.jstor.org/stable/174100}.

\leavevmode\hypertarget{ref-welch1980}{}%
Welch, W. P. 1980. ``The Allocation of Political Monies: Economic
Interest Groups.'' \emph{Public Choice} 35 (1): 97--120.
\url{https://doi.org/10.1007/BF00154752}.

\leavevmode\hypertarget{ref-witko2006}{}%
Witko, Christopher. 2006. ``PACs, Issue Context, and Congressional
Decisionmaking.'' \emph{Political Research Quarterly} 59 (2): 283--95.
\url{https://doi.org/10.1177/106591290605900210}.

\leavevmode\hypertarget{ref-wright1985}{}%
Wright, John R. 1985. ``PACs, Contributions, and Roll Calls: An
Organizational Perspective.'' \emph{American Political Science Review}
79 (2): 400--414. \url{https://doi.org/10.2307/1956656}.

\leavevmode\hypertarget{ref-dezuniga2009}{}%
Zúñiga, Homero Gil De, Eulàlia Puig-I-Abril, and Hernando Rojas. 2009.
``Weblogs, Traditional Sources Online and Political Participation: An
Assessment of How the Internet Is Changing the Political Environment.''
\emph{New Media \& Society} 11 (4): 553--74.
\url{https://doi.org/10.1177/1461444809102960}.





\newpage
\singlespacing 
\end{document}
