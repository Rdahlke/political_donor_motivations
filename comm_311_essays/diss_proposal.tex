\documentclass[12pt,]{article}
\usepackage[left=1in,top=1in,right=1in,bottom=1in]{geometry}
\newcommand*{\authorfont}{\fontfamily{phv}\selectfont}
\usepackage[]{mathpazo}


  \usepackage[T1]{fontenc}
  \usepackage[utf8]{inputenc}




\usepackage{abstract}
\renewcommand{\abstractname}{}    % clear the title
\renewcommand{\absnamepos}{empty} % originally center

\renewenvironment{abstract}
 {{%
    \setlength{\leftmargin}{0mm}
    \setlength{\rightmargin}{\leftmargin}%
  }%
  \relax}
 {\endlist}

\makeatletter
\def\@maketitle{%
  \newpage
%  \null
%  \vskip 2em%
%  \begin{center}%
  \let \footnote \thanks
    {\fontsize{18}{20}\selectfont\raggedright  \setlength{\parindent}{0pt} \@title \par}%
}
%\fi
\makeatother




\setcounter{secnumdepth}{0}







\author{\Large Ross Dahlke\vspace{0.05in} \newline\normalsize\emph{}  }


\date{}

\usepackage{titlesec}

\titleformat*{\section}{\normalsize\bfseries}
\titleformat*{\subsection}{\normalsize\itshape}
\titleformat*{\subsubsection}{\normalsize\itshape}
\titleformat*{\paragraph}{\normalsize\itshape}
\titleformat*{\subparagraph}{\normalsize\itshape}





\newtheorem{hypothesis}{Hypothesis}
\usepackage{setspace}


% set default figure placement to htbp
\makeatletter
\def\fps@figure{htbp}
\makeatother

\usepackage{graphicx}

% move the hyperref stuff down here, after header-includes, to allow for - \usepackage{hyperref}

\makeatletter
\@ifpackageloaded{hyperref}{}{%
\ifxetex
  \PassOptionsToPackage{hyphens}{url}\usepackage[setpagesize=false, % page size defined by xetex
              unicode=false, % unicode breaks when used with xetex
              xetex]{hyperref}
\else
  \PassOptionsToPackage{hyphens}{url}\usepackage[draft,unicode=true]{hyperref}
\fi
}

\@ifpackageloaded{color}{
    \PassOptionsToPackage{usenames,dvipsnames}{color}
}{%
    \usepackage[usenames,dvipsnames]{color}
}
\makeatother
\hypersetup{breaklinks=true,
            bookmarks=true,
            pdfauthor={Ross Dahlke ()},
             pdfkeywords = {},  
            pdftitle={},
            colorlinks=true,
            citecolor=blue,
            urlcolor=blue,
            linkcolor=magenta,
            pdfborder={0 0 0}}
\urlstyle{same}  % don't use monospace font for urls

% Add an option for endnotes. -----


% add tightlist ----------
\providecommand{\tightlist}{%
\setlength{\itemsep}{0pt}\setlength{\parskip}{0pt}}

% add some other packages ----------

% \usepackage{multicol}
% This should regulate where figures float
% See: https://tex.stackexchange.com/questions/2275/keeping-tables-figures-close-to-where-they-are-mentioned
\usepackage[section]{placeins}


\begin{document}
	
% \pagenumbering{arabic}% resets `page` counter to 1 
%




\vskip -8.5pt


 % removetitleabstract

\noindent \doublespacing 

Ross Dahlke

\hypertarget{introduction}{%
\section{Introduction}\label{introduction}}

While the term ``political donor networks'' is often thought of in
abstract terms with a connotation of wealthy political elites with
personal connections to one another, I like to conceive of the term more
literally. The landscape of political donations creates a network
through. We can think of political donors and candidates as nodes who
are connected by donations which serve as edges. Thinking of political
campaigns and political donors as nodes and connected by edges in a
network opens up the rich methodological traditional of social network
analysis to the study of political donors.

Viewing the political donor landscape through a network analysis lens
allows for traditional network statistics to be calculated, such as
centrality, modularity/ polarization, and connection degrees. In
addition, a natural step to take in this network approach is to create
sub-networks of donors. Each cluster or community of donors is comprised
of individuals who have similar network connections. In other words,
donors who are clustered together have similar donations patterns.

Traditional studies of political donors do not take this network
approach. Instead, individuals are treated as the unit of analysis.
Often, behaviors and motivations of political donors are ascertained
through surveys. Researchers often just gather a list of political
donors and send the donors a survey asking them questions about their
demographics and political attitudes. This approach is useful and has
its place, however, it also sacrifices a lot of the rich context and
nuance of the observational data that is generated from political
donations. In addition, individuals donors have very sparse data. It is
difficult to find patterns when an individual only makes one or two
contributions.

Instead, using donor clusters as the unit of analysis maintains the
richness of the contextual information of the network as well as
provides non-sparse data by which one can look for statistically
meaningful patterns. For example, there are two primary theories of the
motivations of political donors, the access-oriented model and the
consumption model. However, it is difficult to measure either of these
behaviors in observational data when taking an individualistic approach.
Instead, using a network approach provides adequate data to test these
models.

The predominant folk-theory of political donors is of smokey backrooms
where donors trade money for favorable votes on legislation. In this
access-oriented model of political donations donors are conceived as
being a \emph{causal} mechanism in legislators taking particular policy
stances. However, the story of political donations since the 2016
election has been of small-dollar, primarily online donors. These donors
have changed the way that political campaigns execute fundraising
operations. This shift in donors lends itself toward the ``consumption''
model of political donations instead of the ``access-oriented'' model.

The consumption model of donors places contributions on a spectrum of
political participation. In other words, donations can be seen as an
extension of voting--a step towards greater participation in democracy.
In this model, donors are \emph{reactive} to politicians. Donors decide
to participate in political campaigns that they already agree with.
These two models, the access-oriented and consumption models, have
conflicting causal orders. Under the access-oriented donor model, donors
cause a change in politicians' policies. In the consumption model,
politicians' policies attract donors.

My research question is: \textbf{Do donations from specific donor
communities impact politicians' public support of policies or does
public support from politicians attract certain political donor
communities?}

Given my previous research findings and a general shift among scholars
towards the consumption model of politics, I theorize that the causal
order is public support for certain policy issues by politicians drives
donations from specific donor communities.

\textbf{H1: Public support of certain policy issues precedes political
donations from various donor clusters.}

The alternative hypothesis, \textbf{H2: Donations from various donor
clusters precedes public support of certain policy issues.}

This question is important to study because it furthers our
understanding of the motivations of groups of political donors--a topic
that has been explored only using an individualistic approach that is
constrained due to various factors. In addition, candidates and
professional political fundraisers spend a significant amount of time
fundraising, but they have little knowledge of the working psychological
process of making a donation.

\hypertarget{time-dimension}{%
\section{Time Dimension}\label{time-dimension}}

Little is definitely known about political donors' motivations, and even
less is known about the temporal dimension of these decisions. However,
thinking of a donation as a decision that seeks to increase one's
preferred candidate's chances of winning can be thought of as being
similar to purchasing a product. There are large brands, companies (ex.
Apple and Microsoft) and parties (Democrats and Republicans), that
provide an immediate heuristic for making a participatory decision.
Then, there is a consumer journey that seeks to build brand awareness
and ultimately triggers a reaction. Under this analogy, just as products
must build awareness and then ultimately trigger a purchase, political
campaigns build awareness of a candidates' policies and try to trigger a
vote or donation. These two different steps in this participatory
process operate on different time scales.

The process to identify as a member of a particular political tribe
takes at least weeks, if not months, years, or even decades. There is a
rich political science literature on political socialization and
generational shifts that show the decades-long timescale of partisan
identification. However, this paper takes a more granular view of
political tribes. The decision to make a political donation to a
particular campaign is like choosing a sub-tribe within one's broader
partisan tribe. Although national figures such as Donald Trump can be a
significant part in speeding up or slowing down shifts that have
multi-year or even multi-decade time horizons, I'm going to constrain my
timescale to be within a single election cycle. Every election cycle has
new candidates running for different offices, generally on different
evolving topics to the point where every election cycle is unique and
generally gets assigned its own ``narrative.'' And so, within an
election cycle, I ask what activates membership within sub-tribes to the
point where an individual makes a contribution to further that
sub-tribe's electoral prospects.

Building this self-identification is similar to firms building brand
awareness. In order for an individual to be primed so that one of the
aforementioned triggers (heightened sense of stakes in the election,
call-to-action, etc.) elicits an action, one must have a sense of
identity or at least support in that sub-tribe. For example, for an
individual who cares about the environment, who may be triggered to make
a contribution to a campaign, must first know that candidate's position
on environmental issues. The length of time it takes campaigns' public
support of certain policy issues to translate into contributions is
largely unstudied. And in part, that is a component of the analysis that
I intend to undertake.

Under the access-oriented model of political donations, where
individuals seek to influence candidates' support of issues. We would
anticipate contributions from members of certain communities resulting
in support of issues later. For example, pro-environment individuals
could make a contributions to a candidate that result in that candidate
being more supportive of conservation policies. Although studies have
been done on the connection between contributions during an election
cycle and legislative votes in the subsequent legislative term, there
has not been much study of the temporal dimension of the possibility of
donations from issue groups manifesting in public support of policy
issues, such as social media posts.

One constraint of this research is the specific event that triggers a
political contribution. Often, decisions to make a donation are
triggered by some discrete event. The specific type of event can vary.
For example, a donor can be triggered to donate in response to news that
alters the perceived stakes of the election, in response to an explicit
solicitation (either by mail, email, or social media), for example at
the end of a fundraising reporting time period, or by attending a
fundraising event, either in-person or increasingly digitally. In the
first two examples, a decision to make a donation is most likely on a
time scale of seconds or milliseconds. The perception of the stakes of
the election increasing, or a response to a call-to-action, is a
psychological response to one's in-group needing assistance. The tribal
nature of our contemporary politics posits donations as a call to arms
to protect that tribe. In the world of marketing sciences, this
last-step mechanism is similar to a ``buy now'' to receive a discount
promotions that is meant to elicit an immediate reaction. While the
last-step mechanism of donating happens within seconds, getting someone
to identify as a member of a political tribe takes longer. My research
does not focus on this last step but the contribution funnel, but it
does acknowledge that more research should be done to understand the
psychological processes that happen in this final step. In addition, my
research might be impacted by these triggering events. For example, one
would expect that contributions may happen around these triggering
events and not at a consistent time from knowledge of a candidate's
public support of a policy issue. These donations around specific events
may cause noise within my data.

While my research is bounded by constraints like long-term political
shifts and short-term donation triggers, it investigates the space
between. I am not entering this research with a precise time that I
expect to be able to measure the connection between public support from
candidates and political donations. There has not been prior research on
the timescale of the process of being primed to make a contribution to a
candidate. This process may happen over days, weeks, or even months. A
part of the methodology that I will use assists in finding this best
time specification to use. And so, this paper will give insight into
both the causal ordering of events (which comes first, political
donations or public support of issues) and insight into the length of
the process that connects those two events.

\hypertarget{where-in-the-communication-or-social-science-literatures}{%
\section{Where in the communication or social science
literatures}\label{where-in-the-communication-or-social-science-literatures}}

This research does not fall cleanly into traditional political
communication traditions or media psychology. Instead, it can find a
home most cleanly within a network science tradition of social science
research.

A lot of political communication research can trace its intellectual
roots back to research conducted on mass communication and propaganda
going back to World War II. This tradition focuses on political actors
disseminating their messages through mass media to various effects.
Although the idea of powerful media effects, such as the hypodermic
needle theory of media, are no longer believed to be true, weaker
modifications, such as agenda setting, remain prevalent. This study does
not fit into this tradition because of its definition of media
(discussed more in-depth later) being social media posts and not
traditional political communication media such as newspaper or
television. While social media posts have the potential to reach a large
audience, they almost never have the scale of print or television news.
In addition, print or television news has an editorial process with
information gatekeepers. These gatekeepers can not only exert editorial
influence over the coverage, but they also choose which information gets
shared in the first place. For example, a candidate could send out a
press release about their support of a policy, but if the news does not
decide to cover the press release, that information is not included in
print or television news. In contrast, on social media, information is
decentralized to where they can encounter a politician's random post
about them supporting a policy issue if they follow the candidate, if
one of their friends shares the post, or a variety of other ways.

The way that media impacts actions, such as political donations, could
fit into a media psychology tradition of communication research. It
would be possible to create a research study that had an experiment
where people were exposed to different social media messages and you
could measure how that exposure impacted their decisions to make
political donations. However, my current study is using observational
data. As discussed previously, an individual-level approach to studying
this research topic has its limitation, including sparsity of data. This
individual-level approach is often taken by those who conduct surveys in
an attempt to study political donor motivations. Generally, a survey
approach to study individual-level behaviors does not include a media
component. Instead, this research study takes a network-based approach
to studying political donors where one donor cluster is the unit of
analysis.

Network sciences within computational social sciences has a strong
intellectual tradition and has seen a recent rise in its use. Mark
Granovetter's seminal work on the strength of weak ties has underpinned
a tradition of social science research that has viewed networks as a
fundamental underpinning of the transmission of information and
behavior. The advent of social media platforms and their inherent
connections and networks that they are based on has given a reemergence
of network sciences. Within contemporary communication and media
research, network science is most often used to study phenomenon on
social media such as retweet networks and echo chamber. Even though this
research project that I am proposing deals with social media, I am using
social media as a layer that goes on top of a network of political
donations. Even though social media networks and political donor
networks are substantively different, methdologies used to study social
media networks can be used for donor networks, such as modularity/
polarization, centrality, and clustering, similar to other social
science network studies such as legislative co-authorship networks in
political science.

\hypertarget{definition-of-media}{%
\section{Definition of media}\label{definition-of-media}}

The definition of media that I am using in this project is each social
media post (Facebook and Twitter) as a unit of media. Each piece of
media is a discrete strategic communication from a campaign to the
public. I inductively hand-coded the topic of 15 percent of the posts
into 26 categories such as liberal on environment issues or conservative
on gun control. I then used these coded posts to classify the remaining
posts using a the BERT deep learning transfer model. Previous political
science literature suggests that political topics can be groups broadly
into liberal social issues, liberal economic issues, conservative social
issues, and conservative economic issues. However, these broader
categories did not have any higher accuracy during classification so I
kept the more granular categories.

Once the posts are classified they can be aggregated in a variety of
ways. For example, you can calculate which topic each campaign posted
most about or which campaigns were most supportive of any topic.

\hypertarget{definition-of-communication}{%
\section{Definition of
communication}\label{definition-of-communication}}

The definition of communication that I am using in this research project
posits communication as political messages expressed by political
campaigns. These communications have both goals and consequences, one
potential consequence being that they attract political donations. Both
the access-oriented donor model and the consumption model of political
donors fall most neatly into a linear model of communication, such as
the Shanon and Weaver model. Both models of political donors can be on
top of this model of communication, but in reverse orders. For example,
under the consumption model of donations, campaigns create a message
around certain policies areas, they disseminate their position on those
issues, for example via social media, which creates a signal that is
received by the potential donor who interprets the message as the
campaign being supportive of the donors' preferred policies or not, and
the donor ultimately deciding whether they should make a donation to
support that campaign or not. This process would be reversed under the
access-oriented donor model where political donors send a message to a
political campaign via their financial support, which causes campaign to
support the policy preferences of the donor. Previous literature has
considered these two models of donors separately. However, there is a
possiblity that both can operate in the same system.

Another possible model of communication is a circular model such as
Osgood and Shramm's model. In this model, participants in the
communication system receive a message, decode the message, interpret
the message, and encode a new message for the other party to decode and
the process continues to circulate messages. This model could
potentially interpret political actors and donors as strategic actors
whose behaviors respond to one another in a game-theoretic way. For
example, a campaign could change its behavior in response to the actions
of political donors, and political donors could change their behavior in
response to campaigns. As it relates to this study, different donor
clusters behave differently under different motivations. For example,
one group of donors could donate to candidates because that candidate
already supports their preferred policies. A different cluster of donors
could contribute to candidates in hopes of gaining access to change
their position on their preferred policy. Different clusters of donors
can be motivated by different processes and outcomes. A circular model,
such as Osgood and Shramm's model of communication leaves room for both
models of political donors to be present but found within different
donor clusters.

\hypertarget{fit-into-current-organization-of-communication-research}{%
\section{Fit into current organization of communication
research}\label{fit-into-current-organization-of-communication-research}}

This research project fits squarely into the political communication
subfield of communication. In this study I am measuring the connections
between two groups of political actors' actions in response to one
another (political candidates and groups of political donors).
Potentially, this research could also fall under the new media subfield,
specifically the study of social media. However, the focus of the paper
would change to be more about how social media is used to accomplish
goals such as attract political donors. Instead, the current focus
treats social media more as a proxy for broader campaign communication
and public support of policy issues.

For conferences, this work would be applicable to either the
International Communication Association Political Communication Division
or the American Political Science Association Political Communication
Division. For journals, this would could be submitted to the Journal of
Politics (where a lot of work on political donors is published),
Political Behavior (more focused on the actions and psychological
processes of the actors involved), or Political Communication (emphasis
on the information/ media ecological component of this research)..

\hypertarget{how-might-this-work-change-in-a-different-field-or-variable}{%
\section{How might this work change in a different field or
variable}\label{how-might-this-work-change-in-a-different-field-or-variable}}

This research is intended to study information ecologies and how
behaviors within political that environment alters the actions of other
actors in the information ecology. While the work on the motivations of
political donor falls within political science, the inclusion of the
information ecologies shifts the work into communication research. For
example, survey-based research into political donor motivations fall
into political science because they treat motivations and beliefs and
static in relativity to media environment. I am able to ground this
research in the field of communication through the observational data
that I have on the information/ media ecology,





\newpage
\singlespacing 
\end{document}
