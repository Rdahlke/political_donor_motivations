\documentclass[12pt,]{article}
\usepackage[left=1in,top=1in,right=1in,bottom=1in]{geometry}
\newcommand*{\authorfont}{\fontfamily{phv}\selectfont}
\usepackage[]{mathpazo}


  \usepackage[T1]{fontenc}
  \usepackage[utf8]{inputenc}




\usepackage{abstract}
\renewcommand{\abstractname}{}    % clear the title
\renewcommand{\absnamepos}{empty} % originally center

\renewenvironment{abstract}
 {{%
    \setlength{\leftmargin}{0mm}
    \setlength{\rightmargin}{\leftmargin}%
  }%
  \relax}
 {\endlist}

\makeatletter
\def\@maketitle{%
  \newpage
%  \null
%  \vskip 2em%
%  \begin{center}%
  \let \footnote \thanks
    {\fontsize{18}{20}\selectfont\raggedright  \setlength{\parindent}{0pt} \@title \par}%
}
%\fi
\makeatother




\setcounter{secnumdepth}{0}







\author{\Large Ross Dahlke\vspace{0.05in} \newline\normalsize\emph{}  }


\date{}

\usepackage{titlesec}

\titleformat*{\section}{\normalsize\bfseries}
\titleformat*{\subsection}{\normalsize\itshape}
\titleformat*{\subsubsection}{\normalsize\itshape}
\titleformat*{\paragraph}{\normalsize\itshape}
\titleformat*{\subparagraph}{\normalsize\itshape}





\newtheorem{hypothesis}{Hypothesis}
\usepackage{setspace}


% set default figure placement to htbp
\makeatletter
\def\fps@figure{htbp}
\makeatother

\usepackage{graphicx}

% move the hyperref stuff down here, after header-includes, to allow for - \usepackage{hyperref}

\makeatletter
\@ifpackageloaded{hyperref}{}{%
\ifxetex
  \PassOptionsToPackage{hyphens}{url}\usepackage[setpagesize=false, % page size defined by xetex
              unicode=false, % unicode breaks when used with xetex
              xetex]{hyperref}
\else
  \PassOptionsToPackage{hyphens}{url}\usepackage[draft,unicode=true]{hyperref}
\fi
}

\@ifpackageloaded{color}{
    \PassOptionsToPackage{usenames,dvipsnames}{color}
}{%
    \usepackage[usenames,dvipsnames]{color}
}
\makeatother
\hypersetup{breaklinks=true,
            bookmarks=true,
            pdfauthor={Ross Dahlke ()},
             pdfkeywords = {},  
            pdftitle={},
            colorlinks=true,
            citecolor=blue,
            urlcolor=blue,
            linkcolor=magenta,
            pdfborder={0 0 0}}
\urlstyle{same}  % don't use monospace font for urls

% Add an option for endnotes. -----


% add tightlist ----------
\providecommand{\tightlist}{%
\setlength{\itemsep}{0pt}\setlength{\parskip}{0pt}}

% add some other packages ----------

% \usepackage{multicol}
% This should regulate where figures float
% See: https://tex.stackexchange.com/questions/2275/keeping-tables-figures-close-to-where-they-are-mentioned
\usepackage[section]{placeins}


\begin{document}
	
% \pagenumbering{arabic}% resets `page` counter to 1 
%




\vskip -8.5pt


 % removetitleabstract

\noindent \doublespacing 

Ross Dahlke

The predominant folk-theory of political donors is of smokey backrooms
where donors trade money for favorable votes on legislation. In this
access-oriented model of political donations donors are conceived as
being a \emph{causal} mechanism in legislators taking particular policy
stances. However, the story of political donations since the 2016
election has been of small-dollar, primarily online donors. These donors
have changed the way that political campaigns execute fundraising
operations. This shift in donors lends itself toward the ``consumption''
model of political donations instead of the ``access-oriented'' model.

The consumption model of donors places contributions on a spectrum of
political participation. In other words, donations can be seen as an
extension of voting--a step towards greater participation in democracy.
In this model, donors are \emph{reactive} to politicians. Donors decide
to participate in political campaigns that they already agree with.
These two models, the access-oriented and consumption models, have
conflicting causal orders. Under the access-oriented donor model, donors
cause a change in politicians' policies. In the consumption model,
politicians' policies attract donors.

My research question is: \textbf{Do donations from specific donor
communities impact politicians' public support of policies or does
public support from politicians attract certain political donor
communities?}

Given my previous research findings and a general shift among scholars
towards the consumption model of politics, I theorize that the causal
order is public support for certain policy issues by politicians drives
donations from specific donor communities.

Little is definitely known about political donors' motivations, and even
less is known about the temporal dimension of these decisions. However,
thinking of a donation as a decision that seeks to increase one's
preferred candidate's chances of winning can be thought of as being
similar to purchasing a product. There are large brands, companies (ex.
Apple and Microsoft) and parties (Democrats and Republicans), that
provide an immediate heuristic for making a participatory decision.
Then, there is a consumer journey that seeks to build brand awareness
and ultimately triggers a reaction. Under this analogy, just as products
must build awareness and then ultimately trigger a purchase, political
campaigns build awareness of a candidates' policies and try to trigger a
vote or donation. These two different steps in this participatory
process operate on different time scales.

The process to identify as a member of a particular political tribe
takes at least weeks, if not months, years, or even decades. There is a
rich political science literature on political socialization and
generational shifts that show the decades-long timescale of partisan
identification. However, this paper takes a more granular view of
political tribes. The decision to make a political donation to a
particular campaign is like choosing a sub-tribe within one's broader
partisan tribe. Although national figures such as Donald Trump can be a
significant part in speeding up or slowing down shifts that have
multi-year or even multi-decade time horizons, I'm going to constrain my
timescale to be within a single election cycle. Every election cycle has
new candidates running for different offices, generally on different
evolving topics to the point where every election cycle is unique and
generally gets assigned its own ``narrative.'' And so, within an
election cycle, I ask what activates membership within sub-tribes to the
point where an individual makes a contribution to further that
sub-tribe's electoral prospects.

Building this self-identification is similar to firms building brand
awareness. In order for an individual to be primed so that one of the
aforementioned triggers (heightened sense of stakes in the election,
call-to-action, etc.) elicits an action, one must have a sense of
identity or at least support in that sub-tribe. For example, for an
individual who cares about the environment, who may be triggered to make
a contribution to a campaign, must first know that candidate's position
on environmental issues. The length of time it takes campaigns' public
support of certain policy issues to translate into contributions is
largely unstudied. And in part, that is a component of the analysis that
I intend to undertake.

Under the access-oriented model of political donations, where
individuals seek to influence candidates' support of issues. We would
anticipate contributions from members of certain communities resulting
in support of issues later. For example, pro-environment individuals
could make a contributions to a candidate that result in that candidate
being more supportive of conservation policies. Although studies have
been done on the connection between contributions during an election
cycle and legislative votes in the subsequent legislative term, there
has not been much study of the temporal dimension of the possibility of
donations from issue groups manifesting in public support of policy
issues, such as social media posts.

One constraint of this research is the specific event that triggers a
political contribution. Often, decisions to make a donation are
triggered by some discrete event. The specific type of event can vary.
For example, a donor can be triggered to donate in response to news that
alters the perceived stakes of the election, in response to an explicit
solicitation (either by mail, email, or social media), for example at
the end of a fundraising reporting time period, or by attending a
fundraising event, either in-person or increasingly digitally. In the
first two examples, a decision to make a donation is most likely on a
time scale of seconds or milliseconds. The perception of the stakes of
the election increasing, or a response to a call-to-action, is a
psychological response to one's in-group needing assistance. The tribal
nature of our contemporary politics posits donations as a call to arms
to protect that tribe. In the world of marketing sciences, this
last-step mechanism is similar to a ``buy now'' to receive a discount
promotions that is meant to elicit an immediate reaction. While the
last-step mechanism of donating happens within seconds, getting someone
to identify as a member of a political tribe takes longer. My research
does not focus on this last step but the contribution funnel, but it
does acknowledge that more research should be done to understand the
psychological processes that happen in this final step. In addition, my
research might be impacted by these triggering events. For example, one
would expect that contributions may happen around these triggering
events and not at a consistent time from knowledge of a candidate's
public support of a policy issue. These donations around specific events
may cause noise within my data.

While my research is bounded by constraints like long-term political
shifts and short-term donation triggers, it investigates the space
between. I am not entering this research with a precise time that I
expect to be able to measure the connection between public support from
candidates and political donations. There has not been prior research on
the timescale of the process of being primed to make a contribution to a
candidate. This process may happen over days, weeks, or even months. A
part of the methodology that I will use assists in finding this best
time specification to use. And so, this paper will give insight into
both the causal ordering of events (which comes first, political
donations or public support of issues) and insight into the length of
the process that connects those two events.





\newpage
\singlespacing 
\end{document}
